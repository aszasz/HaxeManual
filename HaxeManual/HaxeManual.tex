\documentclass{../haxe}

% Translation specifics (avoiding changes to haxe.cls for now)
% language
\usepackage[T1]{fontenc}
\usepackage[main=brazil,english]{babel}
% custom commands
\newcommand\translationextra[1]{#1}
\newcommand\translatenote[1]{\footnote{N.~do T.: #1.}}
% adapted commands from the 'haxe' class
\renewcommand{\define}[4][Definição]
	{\begin{myshaded}\noindent\textbf{#1: #2}\par\nobreak\noindent\ignorespaces#4\label{#3}\end{myshaded}}


% todo-related
\usepackage[left=4.7cm, right=2cm, top=2cm, bottom=4.2cm]{geometry}
\usepackage[draft]{todonotes}
\reversemarginpar

% title (TODO: move this to class file once it looks good)

\renewcommand{\maketitle}{
   \begin{titlepage}
     \setcounter{page}{-1}
			\begin{center}
				~\\[3cm]
				\includegraphics[scale=1.25]{../assets/logo.pdf}~\\[1cm]
				{\huge \bfseries Haxe 3 Manual}\\[7cm]
				Haxe Foundation\\
				\today
			\end{center}
   \end{titlepage}
}


\input{../tikz}

% Conventions:

% run-time, compile-time
% Haxe, Haxelib (unless we are talking about the command itself)
% Haxe Standard Library, Haxe Compiler
% object-oriented

% code example width for ebooks: 47

\begin{document}
\title{Haxe 3 Manual}
\author{Haxe Foundation}
\date{\today}
\maketitle


\clearpage
\todototoc
\listoftodos
\clearpage

\clearpage
\tableofcontents
\clearpage

\chapter{Introdução} % upstream: 31aad54ac93dd33fb16df8470fa77450c17c1497 on Sep 9, 2016 
\label{introduction} %ideally we try to keep the line number in sync with the original file to make reference easier. This is (at this point) a pure translation not a version.
\state{NoContent}

\section{O que é o Haxe}
\label{introduction-what-is-haxe}

\todo{Could we have a big Haxe logo in the First Manual Page (Introduction) under the menu (a bit like a book cover ?) It looks a bit empty now and is a landing page for "Manual"}

Haxe consiste em uma linguagem de programação e um compilador, de alto nível e código aberto. Ele possibilita a compilação de programas, escritos em sintaxe similar a ECMAScript\footnote{http://www.ecma-international.org/publications/standards/Ecma-327.htm}, para múltiplas linguagens de destino(target\translatenote{nos referimos doravante a essas linguagens e plataformas simplesmente por \emph{targets}}). Empregando as abstrações adequadas é possível manter uma única base de código que compila para múltiplos targets.

Haxe é fortemente tipado, mas o sistema de tipagem pode ser subvertido quando preciso. Utilizando informações dos tipos, o sistema de tipagem do Haxe pode detectar erros em tempo de compilação que só seriam percebidos em tempo de execução no target. Mais além, as informações de tipos podem ser usadas pelos geradores de cada target para a criação de código otimizado e robusto.

Atualmente, existem nove targets suportados, que possibilitam diferentes ``casos de uso'':

\begin{center}
\begin{tabular}{| l | l | l |}
	\hline
	Nome & Formato de saída & Usos principais \\ \hline
	Javascript & Sourcecode & Browser, Desktop, Mobile, Server \\
	Neko & Bytecode & Desktop, Server \\
	PHP & Sourcecode & Server \\
	Python & Sourcecode & Desktop, Server \\
	C++ & Sourcecode & Desktop, Mobile, Server \\
	Actionscript 3 & Sourcecode & Browser, Desktop, Mobile \\
	Flash & Bytecode & Browser, Desktop, Mobile \\ 
	Java & Sourcecode & Desktop, Server \\
	C\# & Sourcecode & Desktop, Mobile, Server \\ \hline
\end{tabular}
\end{center}

O restante da seção \ref{introduction} dá uma breve visão geral de com o que um programa em Haxe se parece e como o Haxe evoluiu desde de sua criação em 2005.

\Fullref{types} introduz as sete espécies de tipos em Haxe e discute como eles interagem uns com os outros. Isso continua em \Fullref{type-system}, onde funcionalidades como \emph{unificação}, \emph{parâmetros de tipo} e \emph{inferência de tipos} são explicadas.

\Fullref{class-field} é todo sobre as estruturas de classes do Haxe e, entre outros tópicos, lida com \emph{propriedades}, \emph{campos expandidos em linha}\translatenote{daqui em diante também referidos por \emph{inlined}} e \emph{funções genéricas}.

Em \Fullref{expression} vemos como fazer com que programas realmente façam algo através do uso de \emph{expressões}.

\Fullref{lf} descreve algumas das funcionalidades de Haxe em detalhe, tais como \emph{casamento de padrões},\translatenote{ou \emph{pattern matching}} \emph{interpolação de strings} e \emph{eliminação de código morto}. Isso conclui a referência para a linguagem Haxe.

Seguimos com o material de  referência para o compilador Haxe, que primeiro trata do básico em \Fullref{compiler-usage} antes de tocar nas funcionalidades mais avançadas em \Fullref{cr-features}. Finalmente nos aventuramos na estimulante terra das \emph{macros de Haxe} em \Fullref{macro} para ver como algumas tarefas comuns podem ser grandiosamente simplificadas.

No capítulo seguinte, \Fullref{std}, exploramos importantes tipos e conceitos da Biblioteca Padrão Haxe. Nós então aprendemos sobre o gerenciador de pacotes Haxelib em \Fullref{haxelib}.

Os tipos abstratos\translatenote{abstracts} em Haxe afastam muitas diferenças entre os targets, mas algumas vezes é importante interagir com diretamente com um target, o que é o assunto de \Fullref{target-details}.

\section{Sobre esse documento}
\label{introduction-about-this-document}

Esse documento é o manual oficial para o Haxe 3. Como tal, não é um tutorial para iniciantes e não ensina programação. Entretanto, os tópicos são grosseiramente concebidos para ser lidos em ordem e há referências para tópicos ``vistos anteriormente'' e a tópicos ``ainda por vir''. Em alguns casos, uma seção anterior faz uso de informação de uma seção adiante se isso simplifica a explicação. Essas referências são ligadas adequadamente e deveria, geralmente, não ser um problema a leitura adiantada sobre outros tópicos.

Usamos muito código fonte para manter uma conexão prática dos materiais teóricos. Esses exemplos de código são quase sempre programas completos que vem com uma função main que possa ser compilada ``como está''. Entretanto, algumas vezes unicamente as partes importantes são exibidas.
O código fonte aparece como este:

\begin{lstlisting}
Código haxe aqui
\end{lstlisting}
Ocasionalmente, nós demonstramos como o código de Haxe é gerado, para o que usualmente exibimos o target \target{Javascript}.

Mais além, definimos um conjunto de termos nesse documento. Predominantemente, isso é feito quando se introduz um novo tipo ou quando um termo é específico a Haxe. Nós não definimos todo novo aspecto que introduzimos, e.g. o que é uma classe, para evitar o inchamento do texto. As definições se parecem com:

\define{Nome definido}{define-definition}{Descrição da definição}

Em uns poucos lugares, esse documento tem caixas de \emph{trívia}. Essas incluem informações ``extra-oficiais'' tais como porque certas decisões foram tomadas durante o desenvolvimento do Haxe ou porque uma funcionalidade em particular foi mudada nas versões anteriores do Haxe. Essa informação é geralmente sem importância e pode ser pulada já que é unicamente dita para transmitir curiosidades:

\trivia{Assunto da Trivia}{Informações históricas sobre o desenvolvimento da linguagem}

\section{Autores e contribuições}
\label{introduction-authors-and-contributions}

A maior parte do conteúdo desse documento foi escrita por Simon Krajewski, enquanto trabalhando para a Haxe Foundation\translatenote{e traduzido para o português por Arthur Szász}. Gostaríamos de agradecer essas pessoas por suas contribuições\translatenote{também a Jonas Malaco: Compilação e edição em português}: 

\begin{itemize}
	\item Dan Korostelev: Conteúdo adicional e edição
	\item Caleb Harpre: Conteúdo adicional e edição
	\item Josefiene Pertosa: Edição
	\item Miha Lunar: Edição
	\item Nicolas Cannasse: Criador do Haxe
\end{itemize}

\subsection{Licença}
\label{introduction-license}

O Manual do Haxe pela \href{http://haxe.org/foundation}{Haxe Foundation} está licenciado sob a \href{http://creativecommons.org/licenses/by/4.0/}{Creative Commons Attribution 4.0 International License}.

Baseado nos trabalhos em \href{https://github.com/HaxeFoundation/HaxeManual}{https://github.com/HaxeFoundation/HaxeManual} e\\ \href{https://github.com/aszasz/Manual\_do\_Haxe}{https://github.com/aszasz/Manual\_do\_Haxe}.

\section{Hello World}
\label{introduction-hello-world}

O programa seguinte imprime ``Hello World'' depois de ser compilado e executado:

\haxe{assets/HelloWorld.hx}
Isso pode ser testado salvando o código acima em um arquivo chamado \ic{Main.hx} e invocando o compilador do Haxe assim: \ic{haxe -main Main --interp}. Ele então gera a seguinte saída: \ic{Main.hx:3: Hello world}. Há diversas coisas para aprender disso:
\todo{This generates the following output: too many 'this'. You may like a passive sentence: the following output will be generated...though this is to be avoided, generally}

\begin{itemize}
	\item Programas de Haxe são salvos em arquivos com extensão \ic{.hx}
	\item O compilador de haxe é uma ferramenta de linha de comando que pode ser chamada com parâmetros como \ic{-main} e \ic{--interp}
	\item Programas em haxe tem classes (\type{Main}, primeira letra maiúscula), que tem funções (\expr{main}, em minúsculas).
    \item O nome do arquivo contendo a principal classe de Haxe (main) é o mesmo nome da própria classe (nesse caso \type{Main.hx}).
\end{itemize}

\paragraph{Conteúdo relacionado}
\begin{itemize}
	\item \href{http://code.haxe.org/category/beginner/}{Beginner tutorials and examples} em Haxe Code Cookbook.
\end{itemize}

\section{Histórico}
\label{introduction-haxe-history}
\state{Reviewed}

O projeto Haxe começou em 22 de outubro de 2005 pelo desenvolvedor francês \emph{Nicolas Cannasse}, como um sucessor ao popular compilador de ActionScript2 \emph{MTASC} (Motion-Twin Action Script Compiler), de código aberto, e a sua própria linguagem \emph{MTypes}, que era uma experiência com a aplicação de inferência de tipos em uma linguagem orientada a objetos. A paixão de longa data de Nicolas pela concepção de linguagens de programação e o surgimento de novas oportunidades para juntar diferentes tecnologias como parte de seu trabalho desenvolvendo jogos na \emph{Motion-Twin} o levaram a criação de uma linguagem totalmente nova.

Escrita com X maiúsculo naquele tempo, a versão beta de haXe foi lançada em fevereiro de 2006, com os primeiros targets suportados sendo bytecode de AVM\footnote{Adobe Virtual Machine} e bytecode para \emph{Neko}\footnote{http://nekovm.org}, a maquina virtual do próprio Nicolas.

Nicolas Cannasse, quem permanece como líder do projeto do Haxe até esta data, continuou melhorando o Haxe com uma visão clara, levando à subsequente divulgação de Haxe 1.0 em maio de 2006. Essa primeira versão maior veio com suporte a geração de código para \target{Javascript} e já possuía algumas das funcionalidades que definem o Haxe hoje, como a inferência de tipos e a subtipagem estrutural.

Haxe 1 viu diversas versões menores ao longo de dois anos, ganhando \target{Flash AVM2} como target junto com a ferramenta {haxelib} em agosto de 2006 e o target\target{Actionscript 3} em março de 2007. Durante esses meses, houve forte focalização na melhoria da estabilidade, que resultaram em diversas versões resolvendo pequenos bugs.

Haxe 2.0 foi divulgado em julho de 2008, incluindo o target \target{PHP}, cortesia de \emph{Franco Ponticelli}. Um esforço similar de \emph{Hugh Sanderson} levou a adição do target \target{C++} em julho de 2009, junto com a versão 2.04

Assim como com o Haxe 1, o que seguiu foram diversos meses de atualizações de estabilidade. Em janeiro de 2011 a versão 2.07 foi divulgada com o suporte a \emph{macros}. Por volta dessa época, \emph{Bruno Garcia} se juntou a equipe como mantenedor do target \target{Javascript}, que viu vastas melhorias nos lançamentos seguintes: 2.08 e 2.09.

Depois da versão 2.09, \emph{Simon Krajewski} se juntou ao time e o trabalho em direção ao Haxe 3 começou. Além disso, os targets \target{C\#} e \target{Java} de \emph{Cauê Waneck} acharam seus caminhos para dentro dos builds do Haxe. Decidiu-se, então, fazer uma versão final do Haxe 2, que aconteceu em julho de 2012, com a divulgação do Haxe 2.10.

No final de 2012, a chave para o Haxe 3 foi virada e a equipe do Compilador Haxe, agora amparada pela recém-fundada \emph{Haxe Foundation}\footnote{http://haxe-foundation.org}, se focou nesta próxima grande versão. O Haxe 3 foi subsequentemente lançado em maio de 2013.



\part{Language Reference}
\chapter{Tipos}% upstream: commit 79adb3f36faa851206ab38c055e80645cfd73628 on Oct 15, 2016 
\label{types}%ideally we try to keep the line number in sync with the original file to make reference easier. This is (at this point) a pure translation not a version

O Compilador de Haxe emprega um detalhado sistema de tipos que ajuda a detectar erros de tipagem em um programa durante a compilação. Um erro de tipagem é uma operação inválida sobre um dado tipo como a divisão por um String, a tentativa de acesso a um campo de um Integer ou a chamada a uma função com menos (ou mais) argumentos que o necessário.

Em algumas linguagens essa segurança adicional é custosa porque os programadores são forçados a indicar explicitamente os tipos em suas construções sintáticas:

\begin{lstlisting}
var myButton:MySpecialButton = new MySpecialButton(); // As3
MySpecialButton* myButton = new MySpecialButton(); // C++ 
\end{lstlisting}
As indicações explicitas de tipos não são requeridas em Haxe, porque o compilador pode \emph{inferir} o tipo:

\begin{lstlisting}
var myButton = new MySpecialButton(); // Haxe
\end{lstlisting}
Exploraremos a inferência de em detalhes depois em \Fullref{type-system-type-inference}. Por hora, é suficiente dizer que a variável \expr{myButton} no código acima é reconhecida como uma \emph{instância da classe} type{ MySpecialButton}.

O sistema de tipos de Haxe reconhece sete grupos de tipos\translatenote{a palavra type pode aparecer nessa tradução, ao invés de tipo, se facilitar a compreensão. Nesse caso texto dirá algo como ``o type convertido''}:

\begin{description}
 \item[\emph{Instância de classe}:] um objeto de uma dada classe ou interface
 \item[\emph{Instância de enumeração}:] um valor de uma enumeração de Haxe 
 \item[\emph{Estrutura}:] uma estrutura anônima, i.e., uma coleção de campos com nomes
 \item[\emph{Função}:] um tipo composto de vários vários argumentos e um retorno
 \item[\emph{Dinâmico}:] um tipo coringa que é compatível com qualquer tipo
 \item[\emph{Abstrato}:] um tipo no momento de compilação representado outro tipo diferente na execução
 \item[\emph{Monomorfo}:] um tipo desconhecido que deve mais tarde se tornar um tipo diferente
\end{description}

Nos próximos capítulos, cada um desses grupos de tipos será descrito, bem como as relações entre eles.

\define{Tipo composto}{define-compound-type}{Um tipo composto é um tipo que tem subtipos. Isso inclui qualquer tipo com \tref{parâmetros de tipo}{type-system-type-parameters} e as \tref{funções}{types-function}.}

% When generating the .md, the bellow line seems to generate the 
% Uncaught exception - 02-types.tex:85: characters 9086-9087: No match: TNewline
\section{Tipos básicos}
\label{types-basic-types}

Os \emph{tipos básicos} são \type{Bool}, \type{Float} and \type{Int}. Eles podem ser facilmente identificados na sintaxe por valores como


\begin{itemize}
	\item \expr{true} e \expr{false} para \type{Bool},
	\item \expr{1}, \expr{0}, \expr{-1} e \expr{0xFF0000} para \type{Int} e
	\item \expr{1.0}, \expr{0.0}, \expr{-1.0}, \expr{1e10} para \type{Float}.
\end{itemize}

Tipos básicos não são \tref{classes}{types-class-instance} em Haxe. Eles são implementados como \tref{tipos abstratos}{types-abstract} e estão amarrados ao ``gerenciamento de operadores'' interno ao compilador, como descrito nas seções seguintes.

\subsection{Tipos numéricos}
\label{types-numeric-types}

\define[Tipo]{Float}{define-float}{Representa um número em ponto flutuante IEEE com 64 bits e precisão dupla.}

\define[Tipo]{Int}{define-int}{Representa um número inteiro.}
Enquanto todo \type{Int} pode ser usado onde se espera um \type{Float} (\type{Int} \emph{é atribuível a} ou \emph{unifica com} \type{Float}), o contrário não é verdade: a atribuição de um \type{Float} a um \type{Int} pode tirar precisão e não é permitida implicitamente.

\subsection{Overflow}
\label{types-overflow}

Por razões de desempenho, o Compilador de Haxe não define um comportamento particular de overflow\translatenote{a condição a que se refere aqui como overflow é a codição que se atinge quando o resultado de uma operação é maior do que o espaço de memória destinado a guardá-lo, algumas publicações em português utilizam transbordamento e outras estouro, optamos  por manter o termo original overflow, como o fazem outras tantas publicações.}. A responsabilidade de verificar overflows recai sobre a plataforma do target. Aqui se vêem algumas observações especificas do comportamento de overflow:

\begin{description}
 \item[C++, Java, C\#, Neko, Flash:] Inteiros de 32 bits com sinal e comportamento usual de overflow
 \item[PHP, JS, Flash 8:] Não possuem tipo \emph{Int} nativo e haverá perda de precisão quando eles atingirem seu limite de ponto flutuante (2\textsuperscript{52})
\end{description}

Alternativamente, as classes \emph{haxe.Int32} e \emph{haxe.Int64} podem ser usadas para assegurar correto comportamento de overflow em qualquer plataforma, ao custo de eventuais cálculos adicionais.

\subsection{Operadores numéricos}
\label{types-numeric-operators}

\todo{make sure the types are right for inc, dec, negate, and bitwise negate}
\todo{While introducing the different operations, we should include that information as well, including how they differ with the "C" standard, see http://haxe.org/manual/operators}
Esta é a lista de operadores numéricos em Haxe, agrupados por prioridade decrescente.

\begin{center}
\begin{tabular}{| l | l | l | l | l |}
	\hline
	\multicolumn{5}{|c|}{Aritméticos} \\ \hline
	Operador & Operação & Argumento 1 & Argumento 2 & Retorno \\ \hline
	\expr{++}& incremento & \type{Int} & N/A & \type{Int}\\
	& & \type{Float} & N/A & \type{Float}\\
	\expr{--} & decremento & \type{Int} & N/A & \type{Int}\\
	& & \type{Float} & N/A & \type{Float}\\
	\expr{+} & adição & \type{Float} & \type{Float} & \type{Float} \\
	& & \type{Float} & \type{Int} & \type{Float} \\
	& & \type{Int} & \type{Float} & \type{Float} \\
	& & \type{Int} & \type{Int} & \type{Int} \\
	\expr{-} & subtração & \type{Float} & \type{Float} & \type{Float} \\
	& & \type{Float} & \type{Int} & \type{Float} \\
	& & \type{Int} & \type{Float} & \type{Float} \\
	& & \type{Int} & \type{Int} & \type{Int} \\
	\expr{*} & multiplicação & \type{Float} & \type{Float} & \type{Float} \\
	& & \type{Float} & \type{Int} & \type{Float} \\
	& & \type{Int} & \type{Float} & \type{Float} \\
	& & \type{Int} & \type{Int} & \type{Int} \\	
	\expr{/} & divisão & \type{Float} & \type{Float} & \type{Float} \\
	& & \type{Float} & \type{Int} & \type{Float} \\
	& & \type{Int} & \type{Float} & \type{Float} \\
	& & \type{Int} & \type{Int} & \type{Float} \\
	\expr{\%} & módulo & \type{Float} & \type{Float} & \type{Float} \\
	& & \type{Float} & \type{Int} & \type{Float} \\
	& & \type{Int} & \type{Float} & \type{Float} \\
	& & \type{Int} & \type{Int} & \type{Int} \\	 \hline
	\multicolumn{5}{|c|}{Comparação} \\ \hline
	Operador & Operação & Argumento 1 & Argumento 2 & Retorno \\ \hline
	\expr{==} & igual & \type{Float/Int} & \type{Float/Int} & \type{Bool} \\
	\expr{!=} & diferente & \type{Float/Int} & \type{Float/Int} & \type{Bool} \\
	\expr{<} & menor que & \type{Float/Int} & \type{Float/Int} & \type{Bool} \\
	\expr{<=} & menor ou igual a & \type{Float/Int} & \type{Float/Int} & \type{Bool} \\
	\expr{>} & maior que & \type{Float/Int} & \type{Float/Int} & \type{Bool} \\
	\expr{>=} & maior ou igual a & \type{Float/Int} & \type{Float/Int} & \type{Bool} \\ \hline
	\multicolumn{5}{|c|}{Bit-a-bit} \\ \hline
	Operador & Operação & Argumento 1 & Argumento 2 & Retorno \\ \hline
	\expr{\textasciitilde} & negação bit-a-bit & \type{Int} & N/A & \type{Int} \\	
	\expr{\&} & e bit-a-bit & \type{Int} & \type{Int} & \type{Int} \\	
	\expr{|} & ou bit-a-bit & \type{Int} & \type{Int} & \type{Int} \\	
	\expr{\^} & xor bit-a-bit & \type{Int} & \type{Int} & \type{Int} \\	
	\expr{<{}<} & shift à esquerda & \type{Int} & \type{Int} & \type{Int} \\  % FIXME traduzir shift?
	\expr{>{}>} & shift à direita & \type{Int} & \type{Int} & \type{Int} \\  % FIXME disable ligatures for \expr and other tt typesetting
	\expr{>{}>{}>} & shift à direita sem sinal & \type{Int} & \type{Int} & \type{Int} \\ \hline
\end{tabular}
\end{center}

\paragraph{Igualdade}

\emph{Para enumerações(Enums):}
\begin{description}
	\item[Enumerações sem parâmetros] sempre representam os mesmos valores, então \expr{MyEnum.A == MyEnum.A}.
    \item[Enumerações com parâmetros] podem ser comparadas com \expr{a.equals(b)} (que é um forma curta para \expr{Type.enumEquals()}).
\end{description}

\emph{Dinâmicos(Tipo Dynamic)}
Comparações envolvendo ao menos um valor Dynamic não são especificadas e "específicas a plataforma"

\subsection{Bool}
\label{types-bool}

\define[Type]{Bool(Booleano)}{define-bool}{Representa um valor que pode ser ou \emph{true} ou \emph{false} (verdadeiro ou falso)}

Valores do tipo \type{Bool} são uma ocorrência comum em \emph{condicionais} como \tref{\expr{if}}{expression-if} e \tref{\expr{while}}{expression-while}. Os \emph{operadores} seguintes aceitam e retornam valores \type{Bool}:

\begin{itemize}
    \item \expr{\&\&} (e)
	\item \expr{||} (ou)
	\item \expr{!} (não)
\end{itemize}

O Haxe garante que as expressões booleanas compostas são avaliadas da esquerda para a direita e apenas até onde for necessário em tempo de execução. Por exemplo, uma expressão como \expr{A \&\& B}resolverá primeiro \expr{A}, e resolverá \expr{B} apenas se \expr{A} for verdadeira. Da mesma forma, expressões como \expr{A || B} não resolverão B se \expr{A} for verdadeira. Isso é importante em casos como:

\begin{lstlisting}
if (object != null && object.field == 1) { }
\end{lstlisting}

Acessar \expr{object.field} se \expr{object} for \expr{null} levaria a um erro em tempo de execução, mas a verificação de \expr{object!=null} previne isso.




\subsection{Void}
\label{types-void}

\define[Type]{Void}{define-void}{Void indica a ausência de um tipo. É usado para expressar que algo (normalmente uma função) não tem um valor}

\type{Void} é um caso especial no sitema de tipos porque não é em si um tipo. Ele é usado para expressar a ausência de um tipo. Já vimos Void no exemplo inicial ``Hello World'':
\todo{please review, doubled content}

\haxe{assets/HelloWorld.hx}

O tipo função será explorado em detalhe na seção \Fullref{types-function}, mas uma rápida visão é útil aqui: O tipo da função \expr{main} no exemplo anterior é \type{Void-->Void}, que se lê: ``não tem argumentos e nem valor de retorno''.
Haxe não permite campos e variáveis do tipo \type{Void} e reclamará se uma tentativa de os declarar assim for feita:
\todo{review please, sounds weird}

\begin{lstlisting}
// Arguments and variables of type Void
// are not allowed
var x:Void;
\end{lstlisting}



\section{Nulabilidade}
\label{types-nullability}

\define{nullable}{Um tipo em Haxe é considerado \emph{nullable}\translatenote{a pronúncia é nu-la-bol, a tradução seria ``nulável'', mas usaremos o termo em inglês}se \expr{null} é um valor válido para ele}

É comum que linguagens de programação tenham uma única definição clara de nulabilidade. No entanto, o Haxe tem que se comprometer em relação a esse assunto devido a natureza das linguagens dos targets: Enquanto algumas delas permitem e, de fato, padronizam \expr{null} para tudo, outras nem mesmo permitem \expr{null} para certos tipos. Isso gera a necessidade de criar a distinção entre dois tipos de targets:

\define{Target estático}{define-static-target}{Targets estáticos empregam sistemas de tipificação onde \expr{null} é um valor inválido para os tipos básicos. Esse é o caso de \target{Flash}, \target{C++}, \target{Java} and \target{C\#}.}

\define{Target dinâmico}{define-dynamic-target}{Dynamic targets are more lenient with their types and allow \expr{null} values for basic types. This applies to the \target{JavaScript}, \target{PHP}, \target{Neko} and \target{Flash 6-8} targets.}

Não há nada para se preocupar quando se trabalha com \expr{null} sobre targets dinâmicos; no entanto targets estáticos demandam alguma atenção. Para começar, tipos básicos são inicializados com seus valores padrão.
\todo{for starters...please review}

\define{Valores default}{define-default-value}{
	Tipos básicos tem os seguintes valores default em targets estáticos:
	\begin{description}
		\item[\type{Int}:] \expr{0}
		\item[\type{Float}:] \expr{NaN} on \target{Flash}, \expr{0.0} nos outros targets estáticos
		\item[\type{Bool}:] \expr{false}
	\end{description}
}

Como consequência o Compilador de Haxe não permite que se atribua \expr{null} a um tipo básico em targets estáticos. Para conseguir isso, o tipo básico tem que ser envelopado como \type{Null$<$T$>$}:

\begin{lstlisting}
// error em plataformas estáticas
var a:Int = null;
var b:Null<Int> = null; // permitido
\end{lstlisting}

Similarmente, tipos básicos não podem ser comparados a \expr{null}, a não ser que sejam envelopados:

\begin{lstlisting}
var a : Int = 0;
// error on static platforms
if( a == null ) { ... }
var b : Null<Int> = 0;
if( b != null ) { ... } // allowed
\end{lstlisting}

Essa restrição se estende a todas as situações onde acontece a \tref{unificação}{type-system-unification} 

\define[Type]{\expr{Null<T>}}{define-null-t}{Em targets estáticos os tipos \type{Null<Int>}, \type{Null<Float>} and \type{Null<Bool>} podem ser utilizados para permitir \expr{null} como um valor. Em targets dinâmicos esse envelopamento não tem nenhum efeito. \type{Null<T>} também pode ser usados com outros tipos de forma a documentar que null é um valor permitido.}

Ser um valor \expr{null}  é ``escondido'' em \type{Null$<$T$>$} ou \type{Dynamic} e atribuído a um tipo básico. o valor default da plataforma target é utilizado.

\begin{lstlisting}
var n : Null<Int> = null;
var a : Int = n;
trace(a); // 0 em plataformas estáticas 
\end{lstlisting}



\subsection{Argumentos Opcionais e Nulabilidade}
\label{types-nullability-optional-arguments}

Argumentos opcionais também precisam ser levados em conta quando se considera a nulabilidade.

Em especial, deva haver uma disitinção entre argumentos opcionais \emph{nativos} que não são nuláveis e argumentos opcionais específicos do Haxe que podem ser nuláveis. A distinção pode ser feita utilizando o argumento opcional ponto-de-interrogação:

\begin{lstlisting}
// x is a native Int (not nullable)
function foo(x : Int = 0) {...}
// y is Null<Int> (nullable)
function bar( ?y : Int) {...}
// z is also Null<Int>
function opt( ?z : Int = -1) {...}
\end{lstlisting}
\todo{Is there a difference between \type{?y : Int} and \type{y : Null$<$Int$>$} or can you even do the latter? Some more explanation and examples with native optional and Haxe optional arguments and how they relate to nullability would be nice.}

\trivia{Argumentos x Parametros}{Em algumas outras linguagens de programação, \emph{argumentos} e \emph{parâmetros} são termos intercambiáveis. Em Haxe, \emph{argumento} é usado quando nos referimos a métodos e \emph{parâmetro} quando nos referimos a \Fullref{type-system-type-parameters}.}

\section{Instância de Classe}
\label{types-class-instance}

Similar a muitas linguagens orientadas a objeto, classes são a estrutura de dados primária para a maioria dos programas em Haxe. Cada classe de Haxe tem um nome explícito, um caminho específico e zero ou mais campos. Aqui nos focaremos na estrutura geral de classes e em suas relações e deixaremos os detalhes dos campos da classe para \Fullref{class-field}.
\todo{please review future tense}

O seguinte exemplo de código serve de base para o resto dessa seção:

\haxe{assets/Point.hx}

Semanticamente, essa classe representa um ponto discreto em um espaço bidimensional - mas isso não é importante aqui. Vamos, ao invés disso, descrever a estrutura: 

\begin{itemize}
	\item A palavra-chave \expr{class} informa que estamos declarando uma classe.
	\item \type{Point} é o nome da classe e poderia ser qualquer nome de acordo com \tref{regras de identificadores de tipos}{define-identifier}.
	\item Entre chaves \expr{$\left\{\right\}$} estão os campos da classe,
	\item que consistem de dois campos \emph{variáveis} \expr{x} e \expr{y} do tipo \type{Int},
	\item seguidos por um campo de \emph{function} especial, chamada \expr{new}, que é o \emph{constructor} da classe,
	\item bem como uma função normal \expr{toString}.
\end{itemize}
Há um tipo especial em Haxe que é compatível com todas as classes:

\define[Type]{\expr{Class$<$T$>$}}{define-class-t}{Esse tipo é compatível com todos os tipos de classe, o que siginifica que todas as classes (não suas instâncias) podem ser atribuídas a ele. Durante a compilação, \type{Class<T>} é o tipo base comum de todos os tipos de classes. Entretanto, essa relação não é refletida no código gerado.

Esse tipo é útil quando uma API exige que um valor seja \emph{uma} classe, mas nenhuma específica. Isso se aplica a diversos métodos da \tref{Haxe reflection API}{std-reflection}.}

\subsection{Constructor de classe}
\label{types-class-constructor}

Instâncias de classes são criadas pela chamada do constructor de classe - um processo comumente referido por \emph{instanciação}. Outro nome para instâncias de classes é \emph{objeto}. Apesar disso, preferimos o termo instância de classe, por enfatizar a analogia entre classes/instâncias de classes e \tref{enums/instâncias de enums}{types-enum-instance}.

\begin{lstlisting}
var p = new Point(-1, 65);
\end{lstlisting}
Isso proporcionará uma instância da classe i\type{Point}, que foi alocada em uma variável de nome p. O constructor de \type{Point} recebe os dois argumentos \expr{-1} e \expr{65} e os atribui as variáveis de instância \expr{x} e \expr{y} respectivamente (compare sua definição em \Fullref{types-class-instance}). Revisitaremos o significado exato da expressão \expr{new} mais tarde em \ref{expression-new}. Por ora, apenas pensamos nele como o constructor de classe e retornando o objeto apropriado.



\subsection{Herança}
\label{types-class-inheritance}

Classes podem herdar de outras classes, o que em Haxe é indicado pela palavra-chave \expr{extends}:

\haxe{assets/Point3.hx}
Essa relação é comumente descrita como "é um" (\type{Point3} é um \type{Point}): Qualquer instância da classe \type{Point3} é também uma instância da classe \type{Point}. \type{Point} é então chamada de \emp
h{classe mãe} de \type{Point3}, que é a  \emph{classe filha} de \type{Point}. Uma classe pode ter muitas classes filhas, mas unicamente uma classe mãe. O termo ``uma classe mãe da classe X'' normalmente se rerfere à classe mãe direta, o classe mãe da classe mãe por aí vai. \translatornote{confusco como no original}

O código acima é bastante similar ao código original da classe \type{Point}, com dois novos constructs envolvidos:
\begin{itemize}
 \item \expr{extends Point} indica que essa classe herda de classe \type{Point}
 \item \expr{super (x,y)} é a chamada ao constructror da classe mãe, nesse caso \expr{Point.new}
\end{itemize}
Não é necessário para as classes filhas definirem seus próprios constructors, mas se o fizerem, uma chamada a \expr{super()} é obrigatória. Não como outras linguagens orientadas a objeto, essa chamada pode aparecer em qualquer ponto do código do constructor e não tem que ser a primeira expressão.

Uma classe pode sobreescrever \tref{metódos}{class-field-method} de sua classe mãe, o que exige explicitamente a palavra-chave \expr{override}. Os efeitos e restrições do disso são detalhados adiante em  \Fullref{class-field-overriding}.

\subsection{Interfaces}
\label{types-interfaces}

Uma interface pode ser entendida como a assinatura de uma classe porque ela descreve os campos públicos de uma classe. Interfaces não fornecem implementações mas informações estruturais puras:

\begin{lstlisting}
interface Printable {
	public function toString():String;
}
\end{lstlisting}
A sintaxe é similar a de classes, com as seguintes exceções:

\begin{itemize}
	\item a palavra-chave \expr{interface} é usada ao invés da palavra-chave \expr{class} 
	\item funções não tem nenhuma \tref{expressions}{expression}
	\item todo campo deve ter um tipo explícito 
\end{itemize}
Interfaces, não como \tref{structural subtyping}{type-system-structural-subtyping}, descrevem uma \emph{relação estática} entre classes. Uma dada class é considerada como compatível a uma interface unicamente se ela o declarar explicitamente:

\begin{lstlisting}
class Point implements Printable { }
\end{lstlisting}
Aqui, a palavra-chave \expr{implements} indica que \type{Point} tem uma relação "é uma" com \type{Printable}, i.e. cada instância de \type{Point} é também uma instância de \type{Printable}. Enquanto uma classe só pode ter uma classe mãe, ela pode implementar múltiplas interfaces através de múltiplas palavras-chave \expr{implements}:

\begin{lstlisting}
class Point implements Printable
  implements Serializable
\end{lstlisting}

O compilador verifica se a premissa \expr{implements} é válida. Isso é, ele garante que a classe  realmente implementa todos os campos exigidos pela interface. Um campo é considerado implementado se a classe ou alguma de suas classes ancestrais oferece uma implementação.

Campos de interfaces não são limitados a métodos. Eles podem ser variáveis e propriedades também:

\haxe{assets/InterfaceWithVariables.hx}

Interfaces podem estender múltiplas outras interfaces usando a palavra-chave \expr{extends}:
\begin{lstlisting}
interface Debuggable extends Printable extends Serializable
\end{lstlisting}


\trivia{A Sintaxe de Implements}{Versões de Haxe prévias a 3.0 exigiam que múltiplas palavras-chave \expr{implements} fossem separadas por uma vírgula. Nós decidimos aderir ao ``padrão de facto'' do Java e nos livramos da vírgula. Essa foi uma das mudanças de incompatibilidade entre Haxe 2 and 3.}


\section{Instância de Enum}
\label{types-enum-instance}

Haxe oferece poderosos tipos enumeradores (enumerações ou enum). que são na verdade \emph{tipos de dados algébricos} (Algebraic Data Type)\footnote{\url{http://en.wikipedia.org/wiki/Algebraic_data_type}}. Ainda que não possam ter nenhuma  \tref{expressão}{expression}, eles são muito úteis para a descrição de estruturas de dados:

\haxe{assets/Color.hx}
Semantically, this enum describes a color which is either red, green, blue or a specified RGB value. The syntactic structure is as follows:
\begin{itemize}
	\item The keyword \expr{enum} denotes that we are declaring an enum.
	\item \type{Color} is the name of the enum and could be anything conforming to the rules for \tref{type identifiers}{define-identifier}.
	\item Enclosed in curly braces \expr{$\left\{\right\}$} are the \emph{enum constructors},
	\item which are \expr{Red}, \expr{Green} and \expr{Blue} taking no arguments,
	\item as well as \expr{Rgb} taking three \type{Int} arguments named \expr{r}, \expr{g} and \expr{b}.
\end{itemize}
The Haxe type system provides a type which unifies with all enum types:

\define[Type]{\expr{Enum$<$T$>$}}{define-enum-t}{This type is compatible with all enum types. At compile-time, \type{Enum<T>} can be seen as the common base type of all enum types. However, this relation is not reflected in generated code.} 
\todo{Same as in 2.2, what is \type{Enum$<$T$>$} syntax?}

\subsection{Enum Constructor}
\label{types-enum-constructor}

Similar to classes and their constructors, enums provide a way of instantiating them by using one of their constructors. However, unlike classes, enums provide multiple constructors which can easily be used through their name:

\begin{lstlisting}
var a = Red;
var b = Green;
var c = Rgb(255, 255, 0);
\end{lstlisting}
In this code the type of variables \expr{a}, \expr{b} and \expr{c} is \type{Color}. Variable \expr{c} is initialized using the \expr{Rgb} constructor with arguments.
\todo{list arguments}

All enum instances can be assigned to a special type named \type{EnumValue}.

\define[Type]{EnumValue}{define-enumvalue}{EnumValue is a special type which unifies with all enum instances. It is used by the Haxe Standard Library to provide certain operations for all enum instances and can be employed in user-code accordingly in cases where an API requires \emph{an} enum instance, but not a specific one.}

It is important to distinguish enum types and enum constructors, as this example demonstrates:

\haxe{assets/EnumUnification.hx}

If the commented line is uncommented, the program does not compile because \expr{Red} (an enum constructor) cannot be assigned to a variable of type \type{Enum<Color>} (an enum type). The relation is analogous to a class and its instance.

\trivia{Concrete type parameter for \type{Enum$<$T$>$}}{One of the reviewers of this manual was confused about the difference between \type{Color} and \type{Enum<Color>} in the example above. Indeed, using a concrete type parameter there is pointless and only serves the purpose of demonstration. Usually we would omit the type there and let \tref{type inference}{type-system-type-inference} deal with it.

However, the inferred type would be different from \type{Enum<Color>}. The compiler infers a pseudo-type which has the enum constructors as ``fields''. As of Haxe 3.2.0, it is not possible to express this type in syntax but also, it is never necessary to do so.}



\subsection{Using enums}
\label{types-enum-using}

Enums are a good choice if only a finite set of values should be allowed. The individual \tref{constructors}{types-enum-constructor} then represent the allowed variants and enable the compiler to check if all possible values are respected. This can be seen here:

\haxe{assets/Color2.hx}

After retrieving the value of \expr{color} by assigning the return value of \expr{getColor()} to it, a \tref{\expr{switch} expression}{expression-switch} is used to branch depending on the value. The first three cases \expr{Red}, \expr{Green} and \expr{Blue} are trivial and correspond to the constructors of \type{Color} that have no arguments. The final case \expr{Rgb(r, g, b)} shows how the argument values of a constructor can be extracted: they are available as local variables within the case body expression, just as if a \tref{\expr{var} expression}{expression-var} had been used.

Advanced information on using the \expr{switch} expression will be explored later in the section on \tref{pattern matching}{lf-pattern-matching}.


\section{Anonymous Structure}
\label{types-anonymous-structure}

Anonymous structures can be used to group data without explicitly creating a type. The following example creates a structure with two fields \expr{x} and \expr{name}, and initializes their values to \expr{12} and \expr{"foo"} respectively:

\haxe{assets/Structure.hx}
The general syntactic rules follow:

\begin{enumerate}
	\item A structure is enclosed in curly braces \expr{$\left\{\right\}$} and
	\item Has a \emph{comma-separated} list of key-value-pairs.
	\item A \emph{colon} separates the key, which must be a valid \tref{identifier}{define-identifier}, from the value.
	\item\label{valueanytype} The value can be any Haxe expression.
\end{enumerate}
Rule \ref{valueanytype} implies that structures can be nested and complex, e.g.:

\todo{please reformat}

\begin{lstlisting}
var user = {
  name : "Nicolas",
	age : 32,
	pos : [
	  { x : 0, y : 0 },
		{ x : 1, y : -1 }
  ],
};
\end{lstlisting}
Fields of structures, like classes, are accessed using a \emph{dot} (\expr{.}) like so:

\begin{lstlisting}
// get value of name, which is "Nicolas"
user.name;
// set value of age to 33
user.age = 33;
\end{lstlisting}
It is worth noting that using anonymous structures does not subvert the typing system. The compiler ensures that only available fields are accessed, which means the following program does not compile:

\begin{lstlisting}
class Test {
  static public function main() {
    var point = { x: 0.0, y: 12.0 };
    // { y : Float, x : Float } has no field z
    point.z;
  }
}
\end{lstlisting}
The error message indicates that the compiler knows the type of \expr{point}: It is a structure with fields \expr{x} and \expr{y} of type \type{Float}. Since it has no field \expr{z}, the access fails.
The type of \expr{point} is known through \tref{type inference}{type-system-type-inference}, which thankfully saves us from using explicit types for local variables. However, if \expr{point} was a field, explicit typing would be necessary:

\begin{lstlisting}
class Path {
    var start : { x : Int, y : Int };
    var target : { x : Int, y : Int };
    var current : { x : Int, y : Int };
}
\end{lstlisting}
To avoid this kind of redundant type declaration, especially for more complex structures, it is advised to use a \tref{typedef}{type-system-typedef}:

\begin{lstlisting}
typedef Point = { x : Int, y : Int }

class Path {
    var start : Point;
    var target : Point;
    var current : Point;
}
\end{lstlisting}


\subsection{JSON for Structure Values}
\label{types-structure-json}

It is also possible to use \emph{JavaScript Object Notation} for structures by using \emph{string literals} for the keys:

\begin{lstlisting}
var point = { "x" : 1, "y" : -5 };
\end{lstlisting}
While any string literal is allowed, the field is only considered part of the type if it is a valid \tref{Haxe identifier}{define-identifier}. Otherwise, Haxe syntax does not allow expressing access to such a field, and \tref{reflection}{std-reflection} has to be employed through the use of \expr{Reflect.field} and \expr{Reflect.setField}.

\subsection{Class Notation for Structure Types}
\label{types-structure-class-notation}

When defining a structure type, Haxe allows using the same syntax as described in \Fullref{class-field}. The following \tref{typedef}{type-system-typedef} declares a \type{Point} type with variable fields \expr{x} and \expr{y} of type \type{Int}:

\begin{lstlisting}
typedef Point = {
    var x : Int;
    var y : Int;
}
\end{lstlisting}

\subsection{Optional Fields}
\label{types-structure-optional-fields}

\todo{I don't really know how these work yet.}

\subsection{Impact on Performance}
\label{types-structure-performance}

Using structures and, by extension, \tref{structural subtyping}{type-system-structural-subtyping} has no impact on performance when compiling to \tref{dynamic targets}{define-dynamic-target}. However, on \tref{static targets}{define-static-target} a dynamic lookup has to be performed which is typically slower than a static field access.



\section{Function Type}
\label{types-function}

\todo{It seems a bit convoluted explanations. Should we maybe start by "decoding" the meaning of  Void -> Void, then Int -> Bool -> Float, then maybe have samples using \$type}

The function type, along with the \tref{monomorph}{types-monomorph}, is a type which is usually well-hidden from Haxe users, yet present everywhere. We can make it surface by using \expr{\$type}, a special Haxe identifier which outputs the type its expression has during compilation :

\haxe{assets/FunctionType.hx}

There is a strong resemblance between the declaration of function \expr{test} and the output of the first \expr{\$type} expression, yet also a subtle difference:

\begin{itemize}
	\item \emph{Function arguments} are separated by the special arrow token \expr{->} instead of commas, and
	\item the \emph{function return type} appears at the end after another \expr{->}.
\end{itemize}

In either notation it is obvious that the function \expr{test} accepts a first argument of type \type{Int}, a second argument of type \type{String} and returns a value of type \type{Bool}. If a call to this function, such as \expr{test(1, "foo")}, is made within the second \expr{\$type} expression, the Haxe typer checks if \expr{1} can be assigned to \type{Int} and if \expr{"foo"} can be assigned to \type{String}. The type of the call is then equal to the type of the value \expr{test} returns, which is \type{Bool}.

If a function type has other function types as argument or return type, parentheses can be used to group them correctly. For example, \type{Int -> (Int -> Void) -> Void} represents a function which has a first argument of type \type{Int}, a second argument of function type \type{Int -> Void} and a return of \type{Void}.



\subsection{Optional Arguments}
\label{types-function-optional-arguments}

Optional arguments are declared by prefixing an argument identifier with a question mark \expr{?}:

\haxe[label=assets/OptionalArguments.hx]{assets/OptionalArguments.hx}
Function \expr{test} has two optional arguments: \expr{i} of type \type{Int} and \expr{s} of \type{String}. This is directly reflected in the function type output by line 3. 
This example program calls \expr{test} four times and prints its return value.

\begin{enumerate}
	\item The first call is made without any arguments.
	\item The second call is made with a singular argument \expr{1}.
	\item The third call is made with two arguments \expr{1} and \expr{"foo"}.
	\item The fourth call is made with a singular argument \expr{"foo"}.
\end{enumerate}
The output shows that optional arguments which are omitted from the call have a value of \expr{null}. This implies that the type of these arguments must admit \expr{null} as value, which raises the question of its \tref{nullability}{types-nullability}. The Haxe Compiler ensures that optional basic type arguments are nullable by inferring their type as \type{Null<T>} when compiling to a \tref{static target}{define-static-target}.

While the first three calls are intuitive, the fourth one might come as a surprise: It is indeed allowed to skip optional arguments if the supplied value is assignable to a later argument.


\subsection{Default values}
\label{types-function-default-values}

Haxe allows default values for arguments by assigning a \emph{constant value} to them:

\haxe{assets/DefaultValues.hx}
This example is very similar to the one from \Fullref{types-function-optional-arguments}, with the only difference being that the values \expr{12} and \expr{"bar"} are assigned to the function arguments \expr{i} and \expr{s} respectively. The effect is that the default values are used instead of \expr{null} should an argument be omitted from the call.

%TODO: Default values do not imply nullability, even if the value is \expr{null}. 

Default values in Haxe are not part of the type and are not replaced at call-site (unless the function is \tref{inlined}{class-field-inline}, which can be considered as a more typical approach. On some targets the compiler may still pass \expr{null} for omitted argument values and generate code similar to this into the function:
\begin{lstlisting}
	static function test(i = 12, s = "bar") {
		if (i == null) i = 12;
		if (s == null) s = "bar";
		return "i: " +i + ", s: " +s;
	}
\end{lstlisting}
This should be considered in performance-critical code where a solution without default values may sometimes be more viable.




\section{Dynamic}
\label{types-dynamic}

While Haxe has a static type system, this type system can, in effect, be turned off by using the \type{Dynamic} type. A \emph{dynamic value} can be assigned to anything; and anything can be assigned to it. This has several drawbacks:

\begin{itemize}
	\item The compiler can no longer type-check assignments, function calls and other constructs where specific types are expected.
	\item Certain optimizations, in particular when compiling to static targets, can no longer be employed.
	\item Some common errors, e.g. a typo in a field access, can not be caught at compile-time and likely cause an error at runtime.
	\item \Fullref{cr-dce} cannot detect used fields if they are used through \type{Dynamic}.
\end{itemize}
It is very easy to come up with examples where the usage of \type{Dynamic} can cause problems at runtime. Consider compiling the following two lines to a static target:

\begin{lstlisting}
var d:Dynamic = 1;
d.foo;
\end{lstlisting}

Trying to run a compiled program in the Flash Player yields an error \texttt{Property foo not found on Number and there is no default value}. Without \type{Dynamic}, this would have been detected at compile-time.

\trivia{Dynamic Inference before Haxe 3}{The Haxe 3 compiler never infers a type to \type{Dynamic}, so users must be explicit about it. Previous Haxe versions used to infer arrays of mixed types, e.g. \expr{[1, true, "foo"]}, as \type{Array<Dynamic>}. We found that this behavior introduced too many type problems and thus removed it for Haxe 3.}

Use of \type{Dynamic} should be minimized as there are better options in many situations but sometimes it is just practical to use it. Parts of the Haxe \Fullref{std-reflection} API use it and it is sometimes the best option when dealing with custom data structures that are not known at compile-time.

\type{Dynamic} behaves in a special way when being \tref{unified}{type-system-unification} with a \tref{monomorph}{types-monomorph}. Monomorphs are never bound to \type{Dynamic} which can have surprising results in examples such as this:

\haxe{assets/DynamicInferenceIssue.hx}

Although the return type of \expr{Json.parse} is \type{Dynamic}, the type of local variable \expr{json} is not bound to it and remains a monomorph. It is then inferred as an \tref{anonymous structure}{types-anonymous-structure} upon the \expr{json.length} field access, which causes the following \expr{json[0]} array access to fail. In order to avoid this, the variable \expr{json} can be explicitly typed as \type{Dynamic} by using \expr{var json:Dynamic}.

\trivia{Dynamic in the Standard Library}{Dynamic was quite frequent in the Haxe Standard Library before Haxe 3. With the continuous improvements of the Haxe type system the occurrences of Dynamic were reduced over the releases leading to Haxe 3.}

\subsection{Dynamic with Type Parameter}
\label{types-dynamic-with-type-parameter}

\type{Dynamic} is a special type because it allows explicit declaration with and without a \tref{type parameter}{type-system-type-parameters}. If such a type parameter is provided, the semantics described in \Fullref{types-dynamic} are constrained to all fields being compatible with the parameter type:

\begin{lstlisting}
var att : Dynamic<String> = xml.attributes;
// valid, value is a String
att.name = "Nicolas";
// dito (this documentation is quite old)
att.age = "26";
// error, value is not a String
att.income = 0;
\end{lstlisting}


\subsection{Implementing Dynamic}
\label{types-dynamic-implemented}

Classes can \tref{implement}{types-interfaces} \type{Dynamic} and \type{Dynamic$<$T$>$} which enables arbitrary field access. In the former case, fields can have any type, in the latter, they are constrained to be compatible with the parameter type:

\haxe{assets/ImplementsDynamic.hx}

Implementing \type{Dynamic} does not satisfy the requirements of other implemented interfaces. The expected fields still have to be implemented explicitly.

Classes that implement \type{Dynamic} (with or without type parameter) can also utilize a special method named \expr{resolve}. If a \tref{read access}{define-read-access} is made and the field in question does not exist, the \expr{resolve} method is called with the field name as argument:

\haxe{assets/DynamicResolve.hx}



\section{Abstract}
\label{types-abstract}

An abstract type is a type which is actually a different type at run-time. It is a compile-time feature which defines types ``over'' concrete types in order to modify or augment their behavior:

\haxe[firstline=1,lastline=5]{assets/MyAbstract.hx}
We can derive the following from this example:

\begin{itemize}
	\item The keyword \expr{abstract} denotes that we are declaring an abstract type.
	\item \type{AbstractInt} is the name of the abstract and could be anything conforming to the rules for type identifiers.
	\item Enclosed in parenthesis \expr{()} is the \emph{underlying type} \type{Int}.
	\item Enclosed in curly braces \expr{$\left\{\right\}$} are the fields,
	\item which are a constructor function \expr{new} accepting one argument \expr{i} of type \type{Int}.
\end{itemize}

\define{Underlying Type}{define-underlying-type}{The underlying type of an abstract is the type which is used to represent said abstract at runtime. It is usually a concrete (i.e. non-abstract) type but could be another abstract type as well.}

The syntax is reminiscent of classes and the semantics are indeed similar. In fact, everything in the ``body'' of an abstract (that is everything after the opening curly brace) is parsed as class fields. Abstracts may have \tref{method}{class-field-method} fields and non-\tref{physical}{define-physical-field} \tref{property}{class-field-property} fields.

Furthermore, abstracts can be instantiated and used just like classes:

\haxe[firstline=7,lastline=12]{assets/MyAbstract.hx}
As mentioned before, abstracts are a compile-time feature, so it is interesting to see what the above actually generates. A suitable target for this is \target{JavaScript}, which tends to generate concise and clean code. Compiling the above (using \texttt{haxe -main MyAbstract -js myabstract.js}) shows this \target{JavaScript} code:

\begin{lstlisting}
var a = 12;
console.log(a);
\end{lstlisting}
The abstract type \type{Abstract} completely disappeared from the output and all that is left is a value of its underlying type, \type{Int}. This is because the constructor of \type{Abstract} is inlined - something we shall learn about later in the section \Fullref{class-field-inline} - and its inlined expression assigns a value to \expr{this}. This might be surprising when thinking in terms of classes. However, it is precisely what we want to express in the context of abstracts. Any \emph{inlined member method} of an abstract can assign to \expr{this}, and thus modify the ``internal value''.


A good question at this point is ``What happens if a member function is not declared inline'' because the code obviously has to go somewhere. Haxe creates a private class, known to be the \emph{implementation class}, which has all the abstract member functions as static functions accepting an additional first argument \expr{this} of the underlying type. While technically this is an implementation detail, it can be used for \tref{selective functions}{types-abstract-selective-functions}.



\trivia{Basic Types and abstracts}{Before the advent of abstract types, all basic types were implemented as extern classes or enums. While this nicely took care of some aspects such as \type{Int} being a ``child class'' of \type{Float}, it caused issues elsewhere. For instance, with \type{Float} being an extern class, it would unify with the empty structure \expr{\{\}}, making it impossible to constrain a type to accepting only real objects.}




\subsection{Implicit Casts}
\label{types-abstract-implicit-casts}

Unlike classes, abstracts allow defining implicit casts. There are two kinds of implicit casts:

\begin{description}
	\item[Direct:] Allows direct casting of the abstract type to or from another type. This is defined by adding \expr{to} and \expr{from} rules to the abstract type and is only allowed for types which unify with the underlying type of the abstract.
	\item[Class field:] Allows casting via calls to special cast functions. These functions are defined using \expr{@:to} and \expr{@:from} metadata. This kind of cast is allowed for all types.
\end{description}
The following code example shows an example of \emph{direct} casting:

\haxe{assets/ImplicitCastDirect.hx}
We declare \type{MyAbstract} as being \expr{from Int} and \expr{to Int}, meaning it can be assigned from \type{Int} and assigned to \type{Int}. This is shown in lines 9 and 10, where we first assign the \type{Int} \expr{12} to variable \expr{a} of type \type{MyAbstract} (this works due to the \expr{from Int} declaration) and then that abstract back to variable \expr{b} of type \type{Int} (this works due to the \expr{to Int} declaration).

Class field casts have the same semantics, but are defined completely differently:

\haxe{assets/ImplicitCastField.hx}
By adding \expr{@:from} to a static function, that function qualifies as implicit cast function from its argument type to the abstract. These functions must return a value of the abstract type. They must also be declared \expr{static}.

Similarly, adding \expr{@:to} to a function qualifies it as implicit cast function from the abstract to its return type. These functions are typically member-functions but they can be made \expr{static} and then serve as \tref{selective function}{types-abstract-selective-functions}.

In the example the method \expr{fromString} allows the assignment of value \expr{"3"} to variable \expr{a} of type \type{MyAbstract} while the method \expr{toArray} allows assigning that abstract to variable \expr{b} of type \type{Array<Int>}.

When using this kind of cast, calls to the cast-functions are inserted where required. This becomes obvious when looking at the \target{JavaScript} output:

\begin{lstlisting}
var a = _ImplicitCastField.MyAbstract_Impl_.fromString("3");
var b = _ImplicitCastField.MyAbstract_Impl_.toArray(a);
\end{lstlisting}
This can be further optimized by \tref{inlining}{class-field-inline} both cast functions, turning the output into the following:
\todo{please review your use of ``this'' and try to vary somewhat to avoid too much word repetition}

\begin{lstlisting}
var a = Std.parseInt("3");
var b = [a];
\end{lstlisting}
The \emph{selection algorithm} when assigning a type \expr{A} to a type \expr{B} with at least one of them being an abstract is simple:

\begin{enumerate}
	\item If \expr{A} is not an abstract, go to 3.
	\item If \expr{A} defines a \emph{to}-conversions that admits \expr{B}, go to 6.
	\item If \expr{B} is not an abstract, go to 5.
	\item If \expr{B} defines a \emph{from}-conversions that admits \expr{A}, go to 6.
	\item Stop, unification fails.
	\item Stop, unification succeeds.
\end{enumerate}

\input{assets/tikz/abstract-selection.tex}

By design, implicit casts are \emph{not transitive}, as the following example shows:

\haxe{assets/ImplicitTransitiveCast.hx}
While the individual casts from \type{A} to \type{B} and from \type{B} to \type{C} are allowed, a transitive cast from \type{A} to \type{C} is not. This is to avoid ambiguous cast-paths and retain a simple selection algorithm. 




\subsection{Operator Overloading}
\label{types-abstract-operator-overloading}

Abstracts allow overloading of unary and binary operators by adding the \expr{@:op} metadata to class fields:

\haxe{assets/AbstractOperatorOverload.hx}
By defining \expr{@:op(A * B)}, the function \expr{repeat} serves as operator method for the multiplication \expr{*} operator when the type of the left value is \type{MyAbstract} and the type of the right value is \type{Int}. The usage is shown in line 17, which turns into this when compiled to \target{JavaScript}:

\begin{lstlisting}
console.log(_AbstractOperatorOverload.
  MyAbstract_Impl_.repeat(a,3));
\end{lstlisting}
Similar to \tref{implicit casts with class fields}{types-abstract-implicit-casts}, a call to the overload method is inserted where required.

The example \expr{repeat} function is not commutative: While \expr{MyAbstract * Int} works, \expr{Int * MyAbstract} does not. If this should be allowed as well, the \expr{@:commutative} metadata can be added. If it should work \emph{only} for \expr{Int * MyAbstract}, but not for \expr{MyAbstract * Int}, the overload method can be made static, accepting \type{Int} and \type{MyAbstract} as first and second type respectively.

Overloading unary operators is analogous:

\haxe{assets/AbstractUnopOverload.hx}
Both binary and unary operator overloads can return any type.

\paragraph{Exposing underlying type operations}

It is also possible to omit the method body of a \expr{@:op} function, but only if the underlying type of the abstract allows the operation in question and if the resulting type can be assigned back to the abstract.

\haxe{assets/AbstractExposeTypeOperations.hx}

\todo{please review for correctness}


\subsection{Array Access}
\label{types-abstract-array-access}

Array access describes the particular syntax traditionally used to access the value in an array at a certain offset. This is usually only allowed with arguments of type \type{Int}. Nevertheless, with abstracts it is possible to define custom array access methods. The \tref{Haxe Standard Library}{std} uses this in its \type{Map} type, where the following two methods can be found:
\todo{You have marked ``Map'' for some reason}

\begin{lstlisting}
@:arrayAccess
public inline function get(key:K) {
  return this.get(key);
}
@:arrayAccess
public inline function arrayWrite(k:K, v:V):V {
	this.set(k, v);
	return v;
}
\end{lstlisting}
There are two kinds of array access methods:

\begin{itemize}
	\item If an \expr{@:arrayAccess} method accepts one argument, it is a getter.
	\item If an \expr{@:arrayAccess} method accepts two arguments, it is a setter.
\end{itemize}
The methods \expr{get} and \expr{arrayWrite} seen above then allow the following usage:

\haxe{assets/AbstractArrayAccess.hx}

At this point it should not be surprising to see that calls to the array access fields are inserted in the output:

\begin{lstlisting}
map.set("foo",1);
console.log(map.get("foo")); // 1
\end{lstlisting}

\paragraph{Order of array access resolving}
\label{types-abstract-array-access-order}

Due to a bug in Haxe versions before 3.2 the order of checked \expr{:arrayAccess} fields was undefined. This was fixed for Haxe 3.2 so that the fields are now consistently checked from top to bottom:

\haxe{assets/AbstractArrayAccessOrder.hx}

The array access \expr{a[0]} is resolved to the \expr{getInt1} field, leading to lower case \expr{f} being returned. The result might be different in Haxe versions before 3.2.

Fields which are defined earlier take priority even if they require an \tref{implicit cast}{types-abstract-implicit-casts}.


\subsection{Selective Functions}
\label{types-abstract-selective-functions}

Since the compiler promotes abstract member functions to static functions, it is possible to define static functions by hand and use them on an abstract instance. The semantics here are similar to those of \tref{static extensions}{lf-static-extension}, where the type of the first function argument determines for which types a function is defined:

\haxe{assets/SelectiveFunction.hx}
The method \expr{getString} of abstract \type{MyAbstract} is defined to accept a first argument of \type{MyAbstract$<$String$>$}. This causes it to be available on variable \expr{a} on line 14 (because the type of \expr{a} is \type{MyAbstract$<$String$>$}), but not on variable \expr{b} whose type is \type{MyAbstract$<$Int$>$}.

\trivia{Accidental Feature}{ Rather than having actually been designed, selective functions were discovered. After the idea was first mentioned, it required only minor adjustments in the compiler to make them work. Their discovery also lead to the introduction of multi-type abstracts, such as Map. }


\subsection{Enum abstracts}
\label{types-abstract-enum}
\since{3.1.0}

By adding the \expr{:enum} metadata to an abstract definition, that abstract can be used to define finite value sets:

\haxe{assets/AbstractEnum.hx}

The Haxe Compiler replaces all field access to the \type{HttpStatus} abstract with their values, as evident in the \target{JavaScript} output:

\begin{lstlisting}
Main.main = function() {
	var status = 404;
	var msg = Main.printStatus(status);
};
Main.printStatus = function(status) {
	switch(status) {
	case 404:
		return "Not found";
	case 405:
		return "Method not allowed";
	}
};
\end{lstlisting}

This is similar to accessing \tref{variables declared as inline}{class-field-inline}, but has several advantages:

\begin{itemize}
	\item The typer can ensure that all values of the set are typed correctly.
	\item The pattern matcher checks for \tref{exhaustiveness}{lf-pattern-matching-exhaustiveness} when \tref{matching}{lf-pattern-matching} an enum abstract.
	\item Defining fields requires less syntax.
\end{itemize}


\subsection{Forwarding abstract fields}
\label{types-abstract-forward}
\since{3.1.0}

When wrapping an underlying type, it is sometimes desirable to ``keep'' parts of its functionality. Because writing forwarding functions by hand is cumbersome, Haxe allows adding the \expr{:forward} metadata to an abstract type:

\haxe{assets/AbstractExpose.hx}

The \type{MyArray} abstract in this example wraps \type{Array}. Its \expr{:forward} metadata has two arguments which correspond to the field names to be forwarded to the underlying type. In this example, the \expr{main} method instantiates \type{MyArray} and accesses its \expr{push} and \expr{pop} methods. The commented line demonstrates that the \expr{length} field is not available.

As usual we can look at the \target{JavaScript} output to see how the code is being generated:

\begin{lstlisting}
Main.main = function() {
	var myArray = [];
	myArray.push(12);
	myArray.pop();
};
\end{lstlisting}

It is also possible to use \expr{:forward} without any arguments in order to forward all fields. Of course the Haxe Compiler still ensures that the field actually exists on the underlying type.

\trivia{Implemented as macro}{Both the \expr{:enum} and \expr{:forward} functionality were originally implemented using \tref{build macros}{macro-type-building}. While this worked nicely in non-macro code, it caused issues if these features were used from within macros. The implementation was subsequently moved to the compiler.}


\subsection{Core-type abstracts}
\label{types-abstract-core-type}

The Haxe Standard Library defines a set of basic types as core-type abstracts. They are identified by the \expr{:coreType} metadata and the lack of an underlying type declaration. These abstracts can still be understood to represent a different type. Still, that type is native to the Haxe target. 

Introducing custom core-type abstracts is rarely necessary in user code as it requires the Haxe target to be able to make sense of it. However, there could be interesting use-cases for authors of macros and new Haxe targets.

In contrast to opaque abstracts, core-type abstracts have the following properties:

\begin{itemize}
	\item They have no underlying type.
	\item They are considered nullable unless they are annotated with \expr{:notNull} metadata.
	\item They are allowed to declare \tref{array access}{types-abstract-array-access} functions without expressions.
	\item \tref{Operator overloading fields}{types-abstract-operator-overloading} that have no expression are not forced to adhere to the Haxe type semantics.
\end{itemize}



\section{Monomorph}
\label{types-monomorph}

A monomorph is a type which may, through \tref{unification}{type-system-unification}, morph into a different type later. We shall see details about this type when talking about \tref{type inference}{type-system-type-inference}.

\chapter{Sisstema de tipos}
\label{type-system}

Aprendemos sobre as diversas espécies de tipos em \Fullref{types}  e agora é o momento de ver como eles interagem cuns com os oturos. Nós começarmos suavemente com a introdução de \tref{typedef}{type-system-typedef}, um mecanismo para dar um nome (ou alias) para um tipo mais complexo. Dentre outras coisas, isso virá a calhar quando trabalharmos com tipos que tenham \tref{tipos como parâmetros}{type-system-type-parameters}.

Muito da segurança de tipos é conseguido ao verificar se dois dados tipos dos grupos de tipo em \Fullref{types} sejam compatíveis. Quer dizer, o compilador tenta desempenhar a \emph{unificação} entre eles como detalhado em \Fullref{type-system-unification}.

Todos os tipos são organizados em \emph{módulos} e podem ser referidos através de \emph{paths}. \Fullref{type-system-modules-and-paths} dará uma explicação detalhada sobre a mecânica subjacente.

\section{Typedef}
\label{type-system-typedef}

Nós olhamos rapidamente para typedfs enquanto falávamos sobre \tref{estruturas anôniomas}{types-anonymous-structure} e vimos como poderíamos encurtar uma \tref{tipo structure}{types-anonymous-structure} complexo ao lhe dar um nome. Isso é precisamente para o que typedefs servem bem.  Dar nomes a tipos structure deveria até ser considerado o uso primário deles. De fato, é tão comum que a distinção fica algo nebulosa e muitos usuários do Haxe consideram que typedefs \emph{sejam} realmente o ``structure''.

Um typedef pode dar nome a algum outro tipo:

\begin{lstlisting}
typedef AI = Array<Int>;
\end{lstlisting}
Isso nos capacita a usar \expr{AI} em lugares onde normalmente usaríamos \expr{Array$<$Int$>$}. Enquanto economiza poucas batidas de teclas nesse caso particular, isso pode fazer uma diferença muito maior para casos compostos, mais complexos. Uma vez mais, esse é o porque typedef e structures paracem tão conectados:

\begin{lstlisting}
typedef User = {
  var idade : Int;
  var nome : String;
}
\end{lstlisting}
Um typedef não é uma substituição textual, mas efetivamente um tipo de verdade; Ele pode ainda ter \tref{tipos como parâmetros}{type-system-type-parameters} como o tipo \type{Iterable} da Biblioteca Padrão do Haxe demonstra:

\begin{lstlisting}
typedef Iterable<T> = {
  function iterator() : Iterator<T>;
}
\end{lstlisting}

\paragraph{Campos opcionais}
Registre que o campo de uma estrutura é opcional usando o metadado \ic{@:optional}.
\begin{lstlisting}
typedef User = {
  var age : Int;
  var name : String;
  @:optional var foneNumero : String;
}
\end{lstlisting}

\subsection{Estensões}
\label{type-system-extensions}

% TODO: move to structures? %
Estensões são usadas para expressar que uma estrutura tem todos os campso de um dado tipo, assim como alguns campos adicionais prórios:

\haxe{assets/Extension.hx}
O operador ``maior que'' \expr{>} indica que uma estensão do tipo \type{Iterable$<$T$>$} está sendo criada, como os campos de classe adicionais seguintes. Nesse caso, uma \tref{properiedade}{class-field-property} apenas de leitura  \expr{length} de tipo \type{Int} é necessária.

De forma a ser compatível com \type{IterableWithLength$<$T$>$}, um tipo deve então ser compatível com \type{Iterable$<$T$>$} e também fornecer uma propriedade \expr{length} apenas de leitura do tipo \type{Int}. O exemplo aloca um  \type{Array}, o qual calha de atender essas exigências.

\since{3.1.0}

Também é possível estender multiplas estruturas:

\haxe{assets/Extension2.hx}



\section{Tipos como Parâmetros}
\label{type-system-type-parameters}

Haxe permite a parametrização de uma número considerável de tipos, bem como \tref{campos de classe}{class-field} e \tref{enum constructors}{types-enum-constructor}. Tipos comoParâmetros são definidos por nomes de parâmetros separados por virgulas entre chaves anguladas \expr{$< >$}. Um exemplo simples da Biblioteca Padrão do Haxe é o \type{Array}:

\begin{lstlisting}
class Array<T> {
  function push(x : T) : Int;
}
\end{lstlisting}
Quando quer que uma instância de \type{Array} seja criada, o seu parâmetro \type{T} se torna um \tref{monomorfo}{types-monomorph}. Ou seja, ele pode ser vinculado a qualquer tipo, mas uma única vez. Essa vinculação pode acontecer

\begin{description}
	\item[explicitamente] ao invocar o constructor com tipos explícitos (\expr{new Array$<$String$>$()}) ou
	\item[implicitamente] por \tref{inferência de tipos}{type-system-type-inference}, e.g. quando invocando \expr{instanciaDeArray.push("foo")}.
\end{description}
Dentro da definição de uma classe com tipos como parâmetros, esses ``parâmetros de tipo'' são de um tipo não-especificado. Ao menos que \tref{restrições}{type-system-type-parameter-constraints} sejam adicionadas, o compilador tem que assumir que o ``parâmetro de tipo'' poderia ser usado com qualquer tipo. Como uma consequência, não é possível acessar campos do dos parâmetros de tipo ou \tref{forjar(cast)}{expression-cast} para um tipo que é um ``parâmetro de tipo''. Também não é possível criar uma nova instância de uma tipo que seja um ``parâmetro de tipo'', ao menos que o parâmetro de tipo seja um \tref{tipo genérico}{type-system-generic} e apropriadamente restrito. 

A tabela seguinte mostra onde tipos como parâmetros (``parâmetros de tipo'') são aceitos

\begin{center}
\begin{tabular}{| l | l | l |}
	\hline
	Parâmetro em & Ocasião do vinculamento & Observações \\ \hline
	Classe & instanciamento & Também pode ser vinculado por ocasião de acesso de um campo. \\
	Enum & instanciamento & \\
	Enum Constructor & instanciamento & \\
	Função & chamada & Perimtido para métodos e funções locais lvalue nomeadas. \\
	Structure & instanciamento & \\ \hline
\end{tabular}
\end{center}
Como os tipos de parâmetro de função são vinculados por ocasião da chamada, esse parãmetro de tipo (se irrestrito) aceita qualquer tipo. No entanto, apenas um tipo por chamadaé aceito. Isso pode ser utilizado se uma função tem múltiplos argumentos: 

\haxe{assets/FunctionTypeParameter.hx}

Ambos os argumentos \expr{esperado} and \expr{encontrado} da função \expr{equivale} tem tipo \type{T}. Isso implica que para cada chamada de  \expr{equivale} os dois argumentos devem ser do mesmo tipo. O compilador admite a primeira chamada (ambos argumentos sendo do tipo \type{Int}) e a segunda chamada (ambos argumentos \type{String}), mas a terceira tentativa causa um erro de compilação.

\trivia{``Parâmetros de tipo'' na sintaxe de expressão}{Frequentemente nos perguntam porque um método com parâmetros de tipo não podem ser chamados como \expr{metodo<String>(x)}. As mensagens de erro que o compilador dá não são de muita ajuda. Entretanto, há uma razão simples para isso: O código acima é processado como se ambos \expr{<} and \expr{>} fossem operadores binários, gerando \expr{(metodo < String) > (x)}.}

\subsection{Restrições}
\label{type-system-type-parameter-constraints}

Parâmetros de tipo podem ser restritos a múltiplos tipos:

\haxe{assets/Constraints.hx}
O parâmetro de tipo \type{T} do método \expr{teste} é restrito aos tipos \type{Iterable$<$String$>$} e \type{Mensuravel}. O último é definido usando um \tref{typedef}{type-system-typedef} por conveniência e exige tipos compátiveis que tenham a propriedade \tref{property}{class-field-property} de nome \expr{length} de tipo \type{Int} somente para leitura. As restrições então dizem que um tipo é compatível se

\begin{itemize}
	\item for compatível com  \type{Iterable$<$String$>$} e
	\item tem uma propriedade \expr{length} do tipo \type{Int}.
\end{itemize}
Podemos ver que invocar \expr{teste} com um array vazio na linha 7 e com um \type{Array$<$String$>$} na linha 8 funciona. Isso é porque o tipo \type{Array} tem tanto uma propriedade \expr{length} quanto um método \expr{iterator}. Entretanto, passar um argumento \type{String} na linha 9 falha na verificação de restrição porque \type{String} não é compatível com \type{Iterable$<$T$>$}. 


\section{Generic}
\label{type-system-generic}

Usualmente, o Compilador de Haxe gera unicamente uma classe ou função mesmo que ela tenha parâmetros de tipo. Isso resulta em uma abstração natural onde o gerador de código para a linguagem do target tem que assumir que um parâmetro de tipo pode ser de qualquer tipo. O código gerado então, deveria ter de aplicar algum tipo de verificação de tipo que pode ser detrimental para o desempenho.

Uma classe ou função pode ser feita \emph{genérica} com a atribuição do \tref{metadado}{lf-metadata} \expr{generic}. Isso faz com que o compilador envie uma classe/função distinta para cada combinação de parâmetro de tipo possível com nomes ``decorativos''. Uma especificação como essa pode promover um avanço em porções de código para \tref{targets estáticos}{define-static-target} cujo desempenho é crítico ao custo de arquivos de saída maiores:

\haxe{assets/GenericClass.hx}

Parece pouco usual ver o tipo explícito \type{MyValue<String>} aqui porque usualmente deixamos \tref{a inferência de tipos}{type-system-type-inference} tratar disso. Ainda assim, o explicitamento é de fato necessário nesse caso. O compilador tem que saber o exato tipo de uma classe genérica no momento da construção. A saída em \target{JavaScript} mostra o resultado:

\begin{lstlisting}
(function () { "use strict";
var Test = function() { };
Test.main = function() {
	var a = new MyValue_String("Hello");
	var b = new MyValue_Int(5);
};
var MyValue_Int = function(value) {
	this.value = value;
};
var MyValue_String = function(value) {
	this.value = value;
};
Test.main();
})();
\end{lstlisting}

Podemos identificar que \type{MyValue<String>} e \type{MyValue<Int>} se tornaram \type{MyValue_String} e \type{MyValue_Int} respectivamente. Isso é similar ao que acontece com funções ``generic'':

\haxe{assets/GenericFunction.hx}

De novo, a saída em \target{JavaScript} faz isso óbvio:

\begin{lstlisting}
(function () { "use strict";
var Main = function() { }
Main.method_Int = function(t) {
}
Main.method_String = function(t) {
}
Main.main = function() {
	Main.method_String("foo");
	Main.method_Int(1);
}
Main.main();
})();
\end{lstlisting}


\subsection{Construção de ``parâmetros de tipo'' generic}
\label{type-system-generic-type-parameter-construction}

\define{``Parâmetro de tipo'' generic}{define-generic-type-parameter}{Um parâmetro de tipo é dito ``generic'' se a classe ou método que o contém é `` generic''.}

Não é possível contruir parâmetros de tipo normais, e.g; \expr{new T()} é um erro de compilação. A razão para isso é que Haxe gera somente uma única função e o construct não faz sentido nesse caso. Isso é diferente quando o parâmetro de tipo é genérico: já que sabemos que o compilador vai gerar uma função disitinta para cada combinação de parâmetros de tipo, é possível substituir o \type{T} \expr{new T()} com o tipo real.

\haxe{assets/GenericTypeParameter.hx}

Deve ser observado que a \tref{inferência de cima para baixo}{type-system-top-down-inference} é usada aqui para determinar o tipo efetivo de \type{T}. Há duas exigẽncias para a construção dessa espécie de parâmetro de tipo funcionar: O tipo construído deve ser

\begin{enumerate}
	\item generic e 
	\item explictamente \tref{restrito}{type-system-type-parameter-constraints} para ter um \tref{constructor}{types-class-constructor}.
\end{enumerate}

Here, 1. é dado por \expr{make} contendo o metadado \expr{@:generic}, e 2. por \type{T} sendo restrito a \type{Constructible}. A restrição se mantém tanto para \type{String} quanto para \type{haxe.Template} já que ambos tem um constructor recebendo um único argumento \type{String}. Certo o bastante, a reveladora saída em  \target{JavaScript} se vê conforme esperada:

\begin{lstlisting}
var Main = function() { }
Main.__name__ = true;
Main.make_haxe_Template = function() {
	return new haxe.Template("foo");
}
Main.make_String = function() {
	return new String("foo");
}
Main.main = function() {
	var s = Main.make_String();
	var t = Main.make_haxe_Template();
}
\end{lstlisting}

\section{Variância}
\label{type-system-variance}

Mesmo que variância seja também relevante em outros lugares, ela ocorre especialmente com frequência em parâmetros de tipo e aparece como uma surpresa nesse contexto. Adicionalmente, é muito fácil disparar erros de variância:

\haxe{assets/Variance.hx}

Aparentemente, um \type{Array<Child>} não pode ser alocado a um \type{Array<Base>}, ainda que \type{Child} possa ser alocado a \type{Base}. A razão para isso poderia ser algo inesperada: Isso não é permitido porque se pode escrever em arrays, e.g. via seu método \expr{push()}. É facil causar problemas ao ignorar erros de variância:

\haxe{assets/Variance2.hx}

Aqui, nós subvertemos o verificador de tipos ao usar uma expressão \tref{cast}{expression-cast}, nos  permitindo a atribuição depois da linha comentada. Xom issom now mantivemos uma referência \expr{bases} ao array original, tipado como \type{Array<Base>}. Isso permite inserir outro tipo compatível com \type{Base} (\type{OtherChild}) naquele array. No entanto, nossa referência original  \expr{children} ainda é do tipo \type{Array<Child>} e as coisas vão mal quando encontramos a instância do tipo \type{OtherChild} em um de seus elementos durante uma iteração.

Se \type{Array} não tivesse o método \expr{push()} e nenhuma outra forma de modificação, a alocação seria segura porque nenhum tipo incompátivel poderia ser adicionado a ele. Em Haxe, nós podemos conseguir isso ao restringir o tipo apropriadamente usando \tref{subtipagem estrutural}{type-system-structural-subtyping}:

\haxe{assets/Variance3.hx}

Nós podemos seguramente atribuir com \expr{b} sendo tipada como \type{MyArray<Base>} e \type{MyArray} tendo somente um método \expr{pop()}. Não há método definido em \type{MyArray} que pudesse ser usado para adicionar tipos incompatíveis, ele é assim descrito como \emph{covariante}.

\define{Covariância}{define-covariance}{Um \tref{tipo composto}{define-compound-type} é considerado convariante se seus tipos componentes podem ser atribuídos a componentes menos específicos, i.e. se eles são apenas lidos, mas nunca se escrevem neles.}

\define{Contravariância}{define-contravariance}{Um \tref{tipo composto}{define-compound-type} é considerado contravariante se seus (tipos) componentes podem ser atribuídos a componentes menos genéricos, i.e. se eles são apenas escritos, mas nunca lidos.}






\section{Unification}
\label{type-system-unification}

\todo{Mention toString()/String conversion somewhere in this chapter.}

Unification is the heart of the type system and contributes immensely to the robustness of Haxe programs. It describes the process of checking if a type is compatible to another type.

\define{Unification}{define-unification}{Unification between two types A and B is a directional process which answers the question if A \emph{can be assigned to} B. It may \emph{mutate} either type if it is or has a \tref{monomorph}{types-monomorph}.}

Unification errors are very easy to trigger:

\begin{lstlisting}
class Main {
  static public function main() {
    // Int should be String
    var s:String = 1;
  }
}
\end{lstlisting}
We try to assign a value of type \type{Int} to a variable of type \type{String}, which causes the compiler to try and \emph{unify Int with String}. This is, of course, not allowed and makes the compiler emit the error \expr{Int should be String}.

In this particular case, the unification is triggered by an \emph{assignment}, a context in which the ``is assignable to'' definition is intuitive. It is one of several cases where unification is performed:

\begin{description}
	\item[Assignment:] If \expr{a} is assigned to \expr{b}, the type of \expr{a} is unified with the type of \expr{b}.
	\item[Function call:] We have briefly seen this one while introducing the \tref{function}{types-function} type. In general, the compiler tries to unify the first given argument type with the first expected argument type, the second given argument type with the second expected argument type and so on until all argument types are handled.
	\item[Function return:] Whenever a function has a \expr{return e} expression, the type of \expr{e} is unified with the function return type. If the function has no explicit return type, it is inferred to the type of \expr{e} and subsequent \expr{return} expressions are inferred against it.
	\item[Array declaration:] The compiler tries to find a minimal type between all given types in an array declaration. Refer to \Fullref{type-system-unification-common-base-type} for details.
	\item[Object declaration:] If an object is declared ``against'' a given type, the compiler unifies each given field type with each expected field type.
	\item[Operator unification:] Certain operators expect certain types which the given types are unified against. For instance, the expression \expr{a \&\& b} unifies both \expr{a} and \expr{b} with \type{Bool} and the expression \expr{a == b} unifies \expr{a} with \expr{b}.
\end{description}


\subsection{Between Class/Interface}
\label{type-system-unification-between-classes-and-interfaces}

When defining unification behavior between classes, it is important to remember that unification is directional: We can assign a more specialized class (e.g. a child class) to a generic class (e.g. a parent class) but the reverse is not valid.

The following assignments are allowed:

\begin{itemize}
	\item child class to parent class
	\item class to implementing interface
	\item interface to base interface
\end{itemize}
These rules are transitive, meaning that a child class can also be assigned to the base class of its base class, an interface its base class implements, the base interface of an implementing interface and so on.
\todo{''parent class'' should probably be used here, but I have no idea what it means, so I will refrain from changing it myself.}

\subsection{Structural Subtyping}
\label{type-system-structural-subtyping}

\define{Structural Subtyping}{define-structural-subtyping}{Structural subtyping defines an implicit relation between types that have the same structure.}

Structural sub-typing in Haxe is allowed when unifying

\begin{itemize}
	\item a \tref{class}{types-class-instance} with a \tref{structure}{types-anonymous-structure} and
	\item a structure with another structure.
\end{itemize}

The following example is part of the \type{Lambda} class of the \tref{Haxe Standard Library}{std}:

\begin{lstlisting}
public static function empty<T>(it : Iterable<T>):Bool {
  return !it.iterator().hasNext();
}
\end{lstlisting}
The \expr{empty}-method checks if an \type{Iterable} has an element. For this purpose, it is not necessary to know anything about the argument type other than the fact that it is considered an iterable. This allows calling the \expr{empty}-method with any type that unifies with \type{Iterable$<$T$>$} which applies to a lot of types in the Haxe Standard Library.

This kind of typing can be very convenient but extensive use may be detrimental to performance on static targets, which  is detailed in \Fullref{types-structure-performance}.


\subsection{Monomorphs}
\label{type-system-monomorphs}

Unification of types having or being a \tref{monomorph}{types-monomorph} is detailed in \Fullref{type-system-type-inference}.


\subsection{Function Return}
\label{type-system-unification-function-return}

Unification of function return types may involve the \tref{\type{Void}-type}{types-void} and requires a clear definition of what unifies with \type{Void}. With \type{Void} describing the absence of a type, it is not assignable to any other type, not even \type{Dynamic}. This means that if a function is explicitly declared as returning \type{Dynamic}, it cannot return \type{Void}.

The opposite applies as well: If a function declares a return type of \type{Void}, it cannot return \type{Dynamic} or any other type. However, this direction of unification is allowed when assigning function types:

\begin{lstlisting}
var func:Void->Void = function() return "foo";
\end{lstlisting}

The right-hand function clearly is of type \type{Void->String}, yet we can assign it to the variable \expr{func} of type \type{Void->Void}. This is because the compiler can safely assume that the return type is irrelevant, given that it could not be assigned to any non-\type{Void} type.


\subsection{Common Base Type}
\label{type-system-unification-common-base-type}

Given a set of multiple types, a \emph{common base type} is a type which all types of the set unify against:

\haxe{assets/UnifyMin.hx}
Although \type{Base} is not mentioned, the Haxe Compiler manages to infer it as the common type of \type{Child1} and \type{Child2}. The Haxe Compiler employs this kind of unification in the following situations:

\begin{itemize}
	\item array declarations
	\item \expr{if}/\expr{else}
	\item cases of a \expr{switch}
\end{itemize}




\section{Type Inference}
\label{type-system-type-inference}

The effects of type inference have been seen throughout this document and will continue to be important. A simple example shows type inference at work:

\haxe{assets/TypeInference.hx}
The special construct \expr{\$type} was previously mentioned in order to simplify the explanation of the \Fullref{types-function} type, so let us now introduce it officially:

%TODO: $type
\define[Construct]{\expr{\$type}}{define-dollar-type}{\expr{\$type} is a compile-time mechanism being called like a function, with a single argument. The compiler evaluates the argument expression and then outputs the type of that expression.}

In the example above, the first \expr{\$type} prints \expr{Unknown<0>}. This is a \tref{monomorph}{types-monomorph}, a type that is not yet known. The next line \expr{x = "foo"} assigns a \type{String} literal to \expr{x}, which causes the \tref{unification}{type-system-unification} of the monomorph with \type{String}. We then see that the type of \expr{x} indeed has changed to \type{String}.

Whenever a type other than \Fullref{types-dynamic} is unified with a monomorph, that monomorph \emph{becomes} that type: it \emph{morphs} into that type. Therefore it cannot morph into a different type afterwards, a property expressed in the \emph{mono} part of its name.

Following the rules of unification, type inference can occur in compound types:

\haxe{assets/TypeInference2.hx}
Variable \expr{x} is first initialized to an empty \type{Array}. At this point we can tell that the type of \expr{x} is an array, but we do not yet know the type of the array elements. Consequentially, the type of \expr{x} is \type{Array<Unknown<0>>}. It is only after pushing a \type{String} onto the array that we know the type to be \type{Array<String>}.


\subsection{Top-down Inference}
\label{type-system-top-down-inference}

Most of the time, types are inferred on their own and may then be unified with an expected type. In a few places, however, an expected type may be used to influence inference. We then speak of \emph{top-down inference}.

\define{Expected Type}{define-expected-type}{Expected types occur when the type of an expression is known before that expression has been typed, e.g. because the expression is argument to a function call. They can influence typing of that expression through what is called \tref{top-down inference}{type-system-top-down-inference}.}

A good example are arrays of mixed types. As mentioned in \Fullref{types-dynamic}, the compiler refuses \expr{[1, "foo"]} because it cannot determine an element type. Employing top-down inference, this can be overcome:

\haxe{assets/TopDownInference.hx}

Here, the compiler knows while typing \expr{[1, "foo"]} that the expected type is \type{Array<Dynamic>}, so the element type is \type{Dynamic}. Instead of the usual unification behavior where the compiler would attempt (and fail) to determine a \tref{common base type}{type-system-unification-common-base-type}, the individual elements are typed against and unified with \type{Dynamic}.

We have seen another interesting use of top-down inference when \tref{construction of generic type parameters}{type-system-generic-type-parameter-construction} was introduced:

\haxe{assets/GenericTypeParameter.hx}

The explicit types \type{String} and \type{haxe.Template} are used here to determine the return type of \expr{make}. This works because the method is invoked as \expr{make()}, so we know the return type will be assigned to the variables. Utilizing this information, it is possible to bind the unknown type \type{T} to \type{String} and \type{haxe.Template} respectively.

% this is not really top down inference
%Top-down inference is also utilized when dealing with \tref{enum constructors}{types-enum-constructor}:

%\haxe{assets/TopDownInference2.hx}

%The constructors \expr{TObject} and \expr{TFunction} of type \expr{ValueType} are recognized even though their containing module \type{Type} is not \tref{imported}{Import}. This is possible because the return type of \expr{Type.typeof("foo")} is known to be \expr{ValueType}.


\subsection{Limitations}
\label{type-system-inference-limitations}

Type inference saves a lot of manual type hints when working with local variables, but sometimes the type system still needs some help. In fact, it does not even try to infer the type of a \tref{variable}{class-field-variable} or \tref{property}{class-field-property} field unless it has a direct initialization.

There are also some cases involving recursion where type inference has limitations. If a function calls itself recursively while its type is not (completely) known yet, type inference may infer a wrong, too specialized type.

A different kind of limitation involves the readability of code. If type inference is overused it might be difficult to understand parts of a program due to the lack of visible types. This is particularly true for method signatures. It is recommended to find a good balance between type inference and explicit type hints.


\section{Modules and Paths}
\label{type-system-modules-and-paths}

\define{Module}{define-module}{All Haxe code is organized in modules, which are addressed using paths. In essence, each .hx file represents a module which may contain several types. A type may be \expr{private}, in which case only its containing module can access it.}

The distinction of a module and its containing type of the same name is blurry by design. In fact, addressing \expr{haxe.ds.StringMap<Int>} can be considered shorthand for \expr{haxe.ds.StringMap.StringMap<Int>}. The latter version consists of four parts:

\begin{enumerate}
	\item the package \expr{haxe.ds}
	\item the module name \expr{StringMap}
	\item the type name \type{StringMap}
	\item the type parameter \type{Int}
\end{enumerate}
If the module and type name are equal, the duplicate can be removed, leading to the \expr{haxe.ds.StringMap<Int>} short version. However, knowing about the extended version helps with understanding how \tref{module sub-types}{type-system-module-sub-types} are addressed.

Paths can be shortened further by using an \tref{import}{type-system-import}, which typically allows omitting the package part of a path. This may lead to usage of unqualified identifiers, for which understanding the \tref{resolution order}{type-system-resolution-order} is required.

\define{Type path}{define-type-path}{The (dot-)path to a type consists of the package, the module name and the type name. Its general form is \expr{pack1.pack2.packN.ModuleName.TypeName}.} 


\subsection{Module Sub-Types}
\label{type-system-module-sub-types}

A module sub-type is a type declared in a module with a different name than that module. This allows a single .hx file to contain multiple types, which can be accessed unqualified from within the module, and by using \expr{package.Module.Type} from other modules:

\begin{lstlisting}
var e:haxe.macro.Expr.ExprDef;
\end{lstlisting}

Here the sub-type \type{ExprDef} within module \expr{haxe.macro.Expr} is accessed. 

The sub-type relation is not reflected at run-time. That is, public sub-types become a member of their containing package, which could lead to conflicts if two modules within the same package tried to define the same sub-type. Naturally, the Haxe compiler detects these cases and reports them accordingly. In the example above \type{ExprDef} is generated as \type{haxe.macro.ExprDef}.

Sub-types can also be made private:

\begin{lstlisting}
private class C { ... }
private enum E { ... }
private typedef T { ... }
private abstract A { ... }
\end{lstlisting}

\define{Private type}{define-private-type}{A type can be made private by using the \expr{private} modifier. As a result, the type can only be directly accessed from within the \tref{module}{define-module} it is defined in.

Private types, unlike public ones, do not become a member of their containing package.}

The accessibility of types can be controlled more fine-grained by using \tref{access control}{lf-access-control}.



\subsection{Import}
\label{type-system-import}

If a type path is used multiple times in a .hx file, it might make sense to use an \expr{import} to shorten it. This allows omitting the package when using the type:

\haxe{assets/Import.hx}

With \expr{haxe.ds.StringMap} being imported in the first line, the compiler is able to resolve the unqualified identifier \expr{StringMap} in the \expr{main} function to this package. The module \type{StringMap} is said to be \emph{imported} into the current file.

In this example, we are actually importing a \emph{module}, not just a specific type within that module. This means that all types defined within the imported module are available:

\haxe{assets/Import2.hx}

The type \type{Binop} is an \tref{enum}{types-enum-instance} declared in the module \type{haxe.macro.Expr}, and thus available after the import of said module. If we were to import only a specific type of that module, e.g. \expr{import haxe.macro.Expr.ExprDef}, the program would fail to compile with \expr{Class not found : Binop}.

There are several aspects worth knowing about importing:

\begin{itemize}
	\item The bottommost import takes priority (detailed in \Fullref{type-system-resolution-order}).
	\item The \tref{static extension}{lf-static-extension} keyword \expr{using} implies the effect of \expr{import}.
	\item If an enum is imported (directly or as part of a module import), all its \tref{enum constructors}{types-enum-constructor} are also imported (this is what allows the \expr{OpAdd} usage in the above example).
\end{itemize}

Furthermore, it is also possible to import \tref{static fields}{class-field} of a class and use them unqualified:

\haxe{assets/Import3.hx}

Special care has to be taken with field names or local variable names that conflict with a package name: Since they take priority over packages, a local variable named \expr{haxe} blocks off usage the entire \expr{haxe} package.

\paragraph{Wildcard import}

Haxe allows using \expr{.*} to allow import of all modules in a package, all types in a module or all static fields in a type. It is important to understand that this kind of import only crosses a single level as we can see in the following example:

\haxe{assets/ImportWildcard.hx}

Using the wildcard import on \expr{haxe.macro} allows accessing \type{Expr} which is a module in this package, but it does not allow accessing \type{ExprDef} which is a sub-type of the \type{Expr} module. This rule extends to static fields when a module is imported.

When using wildcard imports on a package the compiler does not eagerly process all modules in that package. This means that these modules are never actually seen by the compiler unless used explicitly and are then not part of the generated output.

\paragraph{Import with alias}

If a type or static field is used a lot in an importing module it might help to alias it to a shorter name. This can also be used to disambiguate conflicting names by giving them a unique identifier.

\haxe{assets/ImportAlias.hx}

Here we import \expr{String.fromCharCode} as \expr{f} which allows us to use \expr{f(65)} and \expr{f(66)}. While the same could be achieved with a local variable, this method is compile-time exclusive and guaranteed to have no run-time overhead.

\since{3.2.0}

Haxe also allows the more natural \expr{as} in place of \expr{in}.


\subsection{Resolution Order}
\label{type-system-resolution-order}

Resolution order comes into play as soon as unqualified identifiers are involved. These are \tref{expressions}{expression} in the form of \expr{foo()}, \expr{foo = 1} and \expr{foo.field}. The last one in particular includes module paths such as \expr{haxe.ds.StringMap}, where \expr{haxe} is an unqualified identifier.  

We describe the resolution order algorithm here, which depends on the following state:

\begin{itemize}
	\item the declared \tref{local variables}{expression-var} (including function arguments)
	\item the \tref{imported}{type-system-import} modules, types and statics
	\item the available \tref{static extensions}{lf-static-extension}
	\item the kind (static or member) of the current field
	\item the declared member fields on the current class and its parent classes
	\item the declared static fields on the current class
	\item the \tref{expected type}{define-expected-type}
	\item the expression being \expr{untyped} or not
\end{itemize}

\todo{proper label and caption + code/identifier styling for diagram}

\input{assets/tikz/resolution-order.tex}

Given an identifier \expr{i}, the algorithm is as follows:

\begin{enumerate}
	\item If i is \expr{true}, \expr{false}, \expr{this}, \expr{super} or \expr{null}, resolve to the matching constant and halt.
	\item If a local variable named \expr{i} is accessible, resolve to it and halt.
	\item If the current field is static, go to \ref{resolution:static-lookup}.
	\item If the current class or any of its parent classes has a field named \expr{i}, resolve to it and halt.
	\item\label{resolution:static-extension} If a static extension with a first argument of the type of the current class is available, resolve to it and halt.
	\item\label{resolution:static-lookup} If the current class has a static field named \expr{i}, resolve to it and halt.
	\item\label{resolution:enum-ctor} If an enum constructor named \expr{i} is declared on an imported enum, resolve to it and halt.
	\item If a static named \expr{i} is explicitly imported, resolve to it and halt.
	\item If \expr{i} starts with a lower-case character, go to \ref{resolution:untyped}.
	\item\label{resolution:type} If a type named \expr{i} is available, resolve to it and halt.
	\item\label{resolution:untyped} If the expression is not in untyped mode, go to \ref{resolution:failure}
	\item If \expr{i} equals \expr{__this__}, resolve to the \expr{this} constant and halt.
	\item Generate a local variable named \expr{i}, resolve to it and halt.
	\item\label{resolution:failure} Fail
\end{enumerate}

For step \ref{resolution:type}, it is also necessary to define the resolution order of types:

\begin{enumerate}
	\item\label{resolution:import} If a type named \expr{i} is imported (directly or as part of a module), resolve to it and halt.
	\item If the current package contains a module named \expr{i} with a type named \expr{i}, resolve to it and halt.
	\item If a type named \expr{i} is available at top-level, resolve to it and halt.
	\item Fail
\end{enumerate}

For step \ref{resolution:import} of this algorithm as well as steps \ref{resolution:static-extension} and \ref{resolution:enum-ctor} of the previous one, the order of import resolution is important:

\begin{itemize}
	\item Imported modules and static extensions are checked from bottom to top with the first match being picked.
	\item Within a given module, types are checked from top to bottom.
	\item For imports, a match is made if the name equals.
	\item For \tref{static extensions}{lf-static-extension}, a match is made if the name equals and the first argument \tref{unifies}{type-system-unification}. Within a given type being used as static extension, the fields are checked from top to bottom.
\end{itemize}

\input{04-class-field.tex}
\input{05-expressions.tex}
\input{06-language-features.tex}

\part{Compiler Reference}
\input{07-compiler-usage.tex}
\input{08-compiler-features.tex}
\input{09-macros.tex}

\part{Standard Library}
\input{10-std.tex}

\part{Miscellaneous}
\input{11-haxelib.tex}
\input{12-target-details.tex}

\end{document}
