\chapter{Sisstema de tipos}
\label{type-system}

Aprendemos sobre as diversas espécies de tipos em \Fullref{types}  e agora é o momento de ver como eles interagem cuns com os oturos. Nós começarmos suavemente com a introdução de \tref{typedef}{type-system-typedef}, um mecanismo para dar um nome (ou alias) para um tipo mais complexo. Dentre outras coisas, isso virá a calhar quando trabalharmos com tipos que tenham \tref{tipos como parâmetros}{type-system-type-parameters}.

Muito da segurança de tipos é conseguido ao verificar se dois dados tipos dos grupos de tipo em \Fullref{types} sejam compatíveis. Quer dizer, o compilador tenta desempenhar a \emph{unificação} entre eles como detalhado em \Fullref{type-system-unification}.

Todos os tipos são organizados em \emph{módulos} e podem ser referidos através de \emph{paths}. \Fullref{type-system-modules-and-paths} dará uma explicação detalhada sobre a mecânica subjacente.

\section{Typedef}
\label{type-system-typedef}

Nós olhamos rapidamente para typedfs enquanto falávamos sobre \tref{estruturas anôniomas}{types-anonymous-structure} e vimos como poderíamos encurtar uma \tref{tipo structure}{types-anonymous-structure} complexo ao lhe dar um nome. Isso é precisamente para o que typedefs servem bem.  Dar nomes a tipos structure deveria até ser considerado o uso primário deles. De fato, é tão comum que a distinção fica algo nebulosa e muitos usuários do Haxe consideram que typedefs \emph{sejam} realmente o ``structure''.

Um typedef pode dar nome a algum outro tipo:

\begin{lstlisting}
typedef AI = Array<Int>;
\end{lstlisting}
Isso nos capacita a usar \expr{AI} em lugares onde normalmente usaríamos \expr{Array$<$Int$>$}. Enquanto economiza poucas batidas de teclas nesse caso particular, isso pode fazer uma diferença muito maior para casos compostos, mais complexos. Uma vez mais, esse é o porque typedef e structures paracem tão conectados:

\begin{lstlisting}
typedef User = {
  var idade : Int;
  var nome : String;
}
\end{lstlisting}
Um typedef não é uma substituição textual, mas efetivamente um tipo de verdade; Ele pode ainda ter \tref{tipos como parâmetros}{type-system-type-parameters} como o tipo \type{Iterable} da Biblioteca Padrão do Haxe demonstra:

\begin{lstlisting}
typedef Iterable<T> = {
  function iterator() : Iterator<T>;
}
\end{lstlisting}

\paragraph{Campos opcionais}
Registre que o campo de uma estrutura é opcional usando o metadado \ic{@:optional}.
\begin{lstlisting}
typedef User = {
  var age : Int;
  var name : String;
  @:optional var foneNumero : String;
}
\end{lstlisting}

\subsection{Estensões}
\label{type-system-extensions}

% TODO: move to structures? %
Estensões são usadas para expressar que uma estrutura tem todos os campso de um dado tipo, assim como alguns campos adicionais prórios:

\haxe{assets/Extension.hx}
O operador ``maior que'' \expr{>} indica que uma estensão do tipo \type{Iterable$<$T$>$} está sendo criada, como os campos de classe adicionais seguintes. Nesse caso, uma \tref{properiedade}{class-field-property} apenas de leitura  \expr{length} de tipo \type{Int} é necessária.

De forma a ser compatível com \type{IterableWithLength$<$T$>$}, um tipo deve então ser compatível com \type{Iterable$<$T$>$} e também fornecer uma propriedade \expr{length} apenas de leitura do tipo \type{Int}. O exemplo aloca um  \type{Array}, o qual calha de atender essas exigências.

\since{3.1.0}

Também é possível estender multiplas estruturas:

\haxe{assets/Extension2.hx}



\section{Tipos como Parâmetros}
\label{type-system-type-parameters}

Haxe permite a parametrização de uma número considerável de tipos, bem como \tref{campos de classe}{class-field} e \tref{enum constructors}{types-enum-constructor}. Tipos comoParâmetros são definidos por nomes de parâmetros separados por virgulas entre chaves anguladas \expr{$< >$}. Um exemplo simples da Biblioteca Padrão do Haxe é o \type{Array}:

\begin{lstlisting}
class Array<T> {
  function push(x : T) : Int;
}
\end{lstlisting}
Quando quer que uma instância de \type{Array} seja criada, o seu parâmetro \type{T} se torna um \tref{monomorfo}{types-monomorph}. Ou seja, ele pode ser vinculado a qualquer tipo, mas uma única vez. Essa vinculação pode acontecer

\begin{description}
	\item[explicitamente] ao invocar o constructor com tipos explícitos (\expr{new Array$<$String$>$()}) ou
	\item[implicitamente] por \tref{inferência de tipos}{type-system-type-inference}, e.g. quando invocando \expr{instanciaDeArray.push("foo")}.
\end{description}
Dentro da definição de uma classe com tipos como parâmetros, esses ``parâmetros de tipo'' são de um tipo não-especificado. Ao menos que \tref{restrições}{type-system-type-parameter-constraints} sejam adicionadas, o compilador tem que assumir que o ``parâmetro de tipo'' poderia ser usado com qualquer tipo. Como uma consequência, não é possível acessar campos do dos parâmetros de tipo ou \tref{forjar(cast)}{expression-cast} para um tipo que é um ``parâmetro de tipo''. Também não é possível criar uma nova instância de uma tipo que seja um ``parâmetro de tipo'', ao menos que o parâmetro de tipo seja um \tref{tipo genérico}{type-system-generic} e apropriadamente restrito. 

A tabela seguinte mostra onde tipos como parâmetros (``parâmetros de tipo'') são aceitos

\begin{center}
\begin{tabular}{| l | l | l |}
	\hline
	Parâmetro em & Ocasião do vinculamento & Observações \\ \hline
	Classe & instanciamento & Também pode ser vinculado por ocasião de acesso de um campo. \\
	Enum & instanciamento & \\
	Enum Constructor & instanciamento & \\
	Função & chamada & Perimtido para métodos e funções locais lvalue nomeadas. \\
	Structure & instanciamento & \\ \hline
\end{tabular}
\end{center}
Como os tipos de parâmetro de função são vinculados por ocasião da chamada, esse parãmetro de tipo (se irrestrito) aceita qualquer tipo. No entanto, apenas um tipo por chamadaé aceito. Isso pode ser utilizado se uma função tem múltiplos argumentos: 

\haxe{assets/FunctionTypeParameter.hx}

Ambos os argumentos \expr{esperado} and \expr{encontrado} da função \expr{equivale} tem tipo \type{T}. Isso implica que para cada chamada de  \expr{equivale} os dois argumentos devem ser do mesmo tipo. O compilador admite a primeira chamada (ambos argumentos sendo do tipo \type{Int}) e a segunda chamada (ambos argumentos \type{String}), mas a terceira tentativa causa um erro de compilação.

\trivia{``Parâmetros de tipo'' na sintaxe de expressão}{Frequentemente nos perguntam porque um método com parâmetros de tipo não podem ser chamados como \expr{metodo<String>(x)}. As mensagens de erro que o compilador dá não são de muita ajuda. Entretanto, há uma razão simples para isso: O código acima é processado como se ambos \expr{<} and \expr{>} fossem operadores binários, gerando \expr{(metodo < String) > (x)}.}

\subsection{Restrições}
\label{type-system-type-parameter-constraints}

Parâmetros de tipo podem ser restritos a múltiplos tipos:

\haxe{assets/Constraints.hx}
O parâmetro de tipo \type{T} do método \expr{teste} é restrito aos tipos \type{Iterable$<$String$>$} e \type{Mensuravel}. O último é definido usando um \tref{typedef}{type-system-typedef} por conveniência e exige tipos compátiveis que tenham a propriedade \tref{property}{class-field-property} de nome \expr{length} de tipo \type{Int} somente para leitura. As restrições então dizem que um tipo é compatível se

\begin{itemize}
	\item for compatível com  \type{Iterable$<$String$>$} e
	\item tem uma propriedade \expr{length} do tipo \type{Int}.
\end{itemize}
Podemos ver que invocar \expr{teste} com um array vazio na linha 7 e com um \type{Array$<$String$>$} na linha 8 funciona. Isso é porque o tipo \type{Array} tem tanto uma propriedade \expr{length} quanto um método \expr{iterator}. Entretanto, passar um argumento \type{String} na linha 9 falha na verificação de restrição porque \type{String} não é compatível com \type{Iterable$<$T$>$}. 


\section{Generic}
\label{type-system-generic}

Usualmente, o Compilador de Haxe gera unicamente uma classe ou função mesmo que ela tenha parâmetros de tipo. Isso resulta em uma abstração natural onde o gerador de código para a linguagem do target tem que assumir que um parâmetro de tipo pode ser de qualquer tipo. O código gerado então, deveria ter de aplicar algum tipo de verificação de tipo que pode ser detrimental para o desempenho.

Uma classe ou função pode ser feita \emph{genérica} com a atribuição do \tref{metadado}{lf-metadata} \expr{generic}. Isso faz com que o compilador envie uma classe/função distinta para cada combinação de parâmetro de tipo possível com nomes ``decorativos''. Uma especificação como essa pode promover um avanço em porções de código para \tref{targets estáticos}{define-static-target} cujo desempenho é crítico ao custo de arquivos de saída maiores:

\haxe{assets/GenericClass.hx}

Parece pouco usual ver o tipo explícito \type{MyValue<String>} aqui porque usualmente deixamos \tref{a inferência de tipos}{type-system-type-inference} tratar disso. Ainda assim, o explicitamento é de fato necessário nesse caso. O compilador tem que saber o exato tipo de uma classe genérica no momento da construção. A saída em \target{JavaScript} mostra o resultado:

\begin{lstlisting}
(function () { "use strict";
var Test = function() { };
Test.main = function() {
	var a = new MyValue_String("Hello");
	var b = new MyValue_Int(5);
};
var MyValue_Int = function(value) {
	this.value = value;
};
var MyValue_String = function(value) {
	this.value = value;
};
Test.main();
})();
\end{lstlisting}

Podemos identificar que \type{MyValue<String>} e \type{MyValue<Int>} se tornaram \type{MyValue_String} e \type{MyValue_Int} respectivamente. Isso é similar ao que acontece com funções ``generic'':

\haxe{assets/GenericFunction.hx}

De novo, a saída em \target{JavaScript} faz isso óbvio:

\begin{lstlisting}
(function () { "use strict";
var Main = function() { }
Main.method_Int = function(t) {
}
Main.method_String = function(t) {
}
Main.main = function() {
	Main.method_String("foo");
	Main.method_Int(1);
}
Main.main();
})();
\end{lstlisting}


\subsection{Construção de ``parâmetros de tipo'' generic}
\label{type-system-generic-type-parameter-construction}

\define{``Parâmetro de tipo'' generic}{define-generic-type-parameter}{Um parâmetro de tipo é dito ``generic'' se a classe ou método que o contém é `` generic''.}

Não é possível contruir parâmetros de tipo normais, e.g; \expr{new T()} é um erro de compilação. A razão para isso é que Haxe gera somente uma única função e o construct não faz sentido nesse caso. Isso é diferente quando o parâmetro de tipo é genérico: já que sabemos que o compilador vai gerar uma função disitinta para cada combinação de parâmetros de tipo, é possível substituir o \type{T} \expr{new T()} com o tipo real.

\haxe{assets/GenericTypeParameter.hx}

Deve ser observado que a \tref{inferência de cima para baixo}{type-system-top-down-inference} é usada aqui para determinar o tipo efetivo de \type{T}. Há duas exigẽncias para a construção dessa espécie de parâmetro de tipo funcionar: O tipo construído deve ser

\begin{enumerate}
	\item generic e 
	\item explictamente \tref{restrito}{type-system-type-parameter-constraints} para ter um \tref{constructor}{types-class-constructor}.
\end{enumerate}

Here, 1. é dado por \expr{make} contendo o metadado \expr{@:generic}, e 2. por \type{T} sendo restrito a \type{Constructible}. A restrição se mantém tanto para \type{String} quanto para \type{haxe.Template} já que ambos tem um constructor recebendo um único argumento \type{String}. Certo o bastante, a reveladora saída em  \target{JavaScript} se vê conforme esperada:

\begin{lstlisting}
var Main = function() { }
Main.__name__ = true;
Main.make_haxe_Template = function() {
	return new haxe.Template("foo");
}
Main.make_String = function() {
	return new String("foo");
}
Main.main = function() {
	var s = Main.make_String();
	var t = Main.make_haxe_Template();
}
\end{lstlisting}

\section{Variância}
\label{type-system-variance}

Mesmo que variância seja também relevante em outros lugares, ela ocorre especialmente com frequência em parâmetros de tipo e aparece como uma surpresa nesse contexto. Adicionalmente, é muito fácil disparar erros de variância:

\haxe{assets/Variance.hx}

Aparentemente, um \type{Array<Child>} não pode ser alocado a um \type{Array<Base>}, ainda que \type{Child} possa ser alocado a \type{Base}. A razão para isso poderia ser algo inesperada: Isso não é permitido porque se pode escrever em arrays, e.g. via seu método \expr{push()}. É facil causar problemas ao ignorar erros de variância:

\haxe{assets/Variance2.hx}

Aqui, nós subvertemos o verificador de tipos ao usar uma expressão \tref{cast}{expression-cast}, nos  permitindo a atribuição depois da linha comentada. Xom issom now mantivemos uma referência \expr{bases} ao array original, tipado como \type{Array<Base>}. Isso permite inserir outro tipo compatível com \type{Base} (\type{OtherChild}) naquele array. No entanto, nossa referência original  \expr{children} ainda é do tipo \type{Array<Child>} e as coisas vão mal quando encontramos a instância do tipo \type{OtherChild} em um de seus elementos durante uma iteração.

Se \type{Array} não tivesse o método \expr{push()} e nenhuma outra forma de modificação, a alocação seria segura porque nenhum tipo incompátivel poderia ser adicionado a ele. Em Haxe, nós podemos conseguir isso ao restringir o tipo apropriadamente usando \tref{subtipagem estrutural}{type-system-structural-subtyping}:

\haxe{assets/Variance3.hx}

Nós podemos seguramente atribuir com \expr{b} sendo tipada como \type{MyArray<Base>} e \type{MyArray} tendo somente um método \expr{pop()}. Não há método definido em \type{MyArray} que pudesse ser usado para adicionar tipos incompatíveis, ele é assim descrito como \emph{covariante}.

\define{Covariância}{define-covariance}{Um \tref{tipo composto}{define-compound-type} é considerado convariante se seus tipos componentes podem ser atribuídos a componentes menos específicos, i.e. se eles são apenas lidos, mas nunca se escrevem neles.}

\define{Contravariância}{define-contravariance}{Um \tref{tipo composto}{define-compound-type} é considerado contravariante se seus (tipos) componentes podem ser atribuídos a componentes menos genéricos, i.e. se eles são apenas escritos, mas nunca lidos.}






\section{Unification}
\label{type-system-unification}

\todo{Mention toString()/String conversion somewhere in this chapter.}

Unification is the heart of the type system and contributes immensely to the robustness of Haxe programs. It describes the process of checking if a type is compatible to another type.

\define{Unification}{define-unification}{Unification between two types A and B is a directional process which answers the question if A \emph{can be assigned to} B. It may \emph{mutate} either type if it is or has a \tref{monomorph}{types-monomorph}.}

Unification errors are very easy to trigger:

\begin{lstlisting}
class Main {
  static public function main() {
    // Int should be String
    var s:String = 1;
  }
}
\end{lstlisting}
We try to assign a value of type \type{Int} to a variable of type \type{String}, which causes the compiler to try and \emph{unify Int with String}. This is, of course, not allowed and makes the compiler emit the error \expr{Int should be String}.

In this particular case, the unification is triggered by an \emph{assignment}, a context in which the ``is assignable to'' definition is intuitive. It is one of several cases where unification is performed:

\begin{description}
	\item[Assignment:] If \expr{a} is assigned to \expr{b}, the type of \expr{a} is unified with the type of \expr{b}.
	\item[Function call:] We have briefly seen this one while introducing the \tref{function}{types-function} type. In general, the compiler tries to unify the first given argument type with the first expected argument type, the second given argument type with the second expected argument type and so on until all argument types are handled.
	\item[Function return:] Whenever a function has a \expr{return e} expression, the type of \expr{e} is unified with the function return type. If the function has no explicit return type, it is inferred to the type of \expr{e} and subsequent \expr{return} expressions are inferred against it.
	\item[Array declaration:] The compiler tries to find a minimal type between all given types in an array declaration. Refer to \Fullref{type-system-unification-common-base-type} for details.
	\item[Object declaration:] If an object is declared ``against'' a given type, the compiler unifies each given field type with each expected field type.
	\item[Operator unification:] Certain operators expect certain types which the given types are unified against. For instance, the expression \expr{a \&\& b} unifies both \expr{a} and \expr{b} with \type{Bool} and the expression \expr{a == b} unifies \expr{a} with \expr{b}.
\end{description}


\subsection{Between Class/Interface}
\label{type-system-unification-between-classes-and-interfaces}

When defining unification behavior between classes, it is important to remember that unification is directional: We can assign a more specialized class (e.g. a child class) to a generic class (e.g. a parent class) but the reverse is not valid.

The following assignments are allowed:

\begin{itemize}
	\item child class to parent class
	\item class to implementing interface
	\item interface to base interface
\end{itemize}
These rules are transitive, meaning that a child class can also be assigned to the base class of its base class, an interface its base class implements, the base interface of an implementing interface and so on.
\todo{''parent class'' should probably be used here, but I have no idea what it means, so I will refrain from changing it myself.}

\subsection{Structural Subtyping}
\label{type-system-structural-subtyping}

\define{Structural Subtyping}{define-structural-subtyping}{Structural subtyping defines an implicit relation between types that have the same structure.}

Structural sub-typing in Haxe is allowed when unifying

\begin{itemize}
	\item a \tref{class}{types-class-instance} with a \tref{structure}{types-anonymous-structure} and
	\item a structure with another structure.
\end{itemize}

The following example is part of the \type{Lambda} class of the \tref{Haxe Standard Library}{std}:

\begin{lstlisting}
public static function empty<T>(it : Iterable<T>):Bool {
  return !it.iterator().hasNext();
}
\end{lstlisting}
The \expr{empty}-method checks if an \type{Iterable} has an element. For this purpose, it is not necessary to know anything about the argument type other than the fact that it is considered an iterable. This allows calling the \expr{empty}-method with any type that unifies with \type{Iterable$<$T$>$} which applies to a lot of types in the Haxe Standard Library.

This kind of typing can be very convenient but extensive use may be detrimental to performance on static targets, which  is detailed in \Fullref{types-structure-performance}.


\subsection{Monomorphs}
\label{type-system-monomorphs}

Unification of types having or being a \tref{monomorph}{types-monomorph} is detailed in \Fullref{type-system-type-inference}.


\subsection{Function Return}
\label{type-system-unification-function-return}

Unification of function return types may involve the \tref{\type{Void}-type}{types-void} and requires a clear definition of what unifies with \type{Void}. With \type{Void} describing the absence of a type, it is not assignable to any other type, not even \type{Dynamic}. This means that if a function is explicitly declared as returning \type{Dynamic}, it cannot return \type{Void}.

The opposite applies as well: If a function declares a return type of \type{Void}, it cannot return \type{Dynamic} or any other type. However, this direction of unification is allowed when assigning function types:

\begin{lstlisting}
var func:Void->Void = function() return "foo";
\end{lstlisting}

The right-hand function clearly is of type \type{Void->String}, yet we can assign it to the variable \expr{func} of type \type{Void->Void}. This is because the compiler can safely assume that the return type is irrelevant, given that it could not be assigned to any non-\type{Void} type.


\subsection{Common Base Type}
\label{type-system-unification-common-base-type}

Given a set of multiple types, a \emph{common base type} is a type which all types of the set unify against:

\haxe{assets/UnifyMin.hx}
Although \type{Base} is not mentioned, the Haxe Compiler manages to infer it as the common type of \type{Child1} and \type{Child2}. The Haxe Compiler employs this kind of unification in the following situations:

\begin{itemize}
	\item array declarations
	\item \expr{if}/\expr{else}
	\item cases of a \expr{switch}
\end{itemize}




\section{Type Inference}
\label{type-system-type-inference}

The effects of type inference have been seen throughout this document and will continue to be important. A simple example shows type inference at work:

\haxe{assets/TypeInference.hx}
The special construct \expr{\$type} was previously mentioned in order to simplify the explanation of the \Fullref{types-function} type, so let us now introduce it officially:

%TODO: $type
\define[Construct]{\expr{\$type}}{define-dollar-type}{\expr{\$type} is a compile-time mechanism being called like a function, with a single argument. The compiler evaluates the argument expression and then outputs the type of that expression.}

In the example above, the first \expr{\$type} prints \expr{Unknown<0>}. This is a \tref{monomorph}{types-monomorph}, a type that is not yet known. The next line \expr{x = "foo"} assigns a \type{String} literal to \expr{x}, which causes the \tref{unification}{type-system-unification} of the monomorph with \type{String}. We then see that the type of \expr{x} indeed has changed to \type{String}.

Whenever a type other than \Fullref{types-dynamic} is unified with a monomorph, that monomorph \emph{becomes} that type: it \emph{morphs} into that type. Therefore it cannot morph into a different type afterwards, a property expressed in the \emph{mono} part of its name.

Following the rules of unification, type inference can occur in compound types:

\haxe{assets/TypeInference2.hx}
Variable \expr{x} is first initialized to an empty \type{Array}. At this point we can tell that the type of \expr{x} is an array, but we do not yet know the type of the array elements. Consequentially, the type of \expr{x} is \type{Array<Unknown<0>>}. It is only after pushing a \type{String} onto the array that we know the type to be \type{Array<String>}.


\subsection{Top-down Inference}
\label{type-system-top-down-inference}

Most of the time, types are inferred on their own and may then be unified with an expected type. In a few places, however, an expected type may be used to influence inference. We then speak of \emph{top-down inference}.

\define{Expected Type}{define-expected-type}{Expected types occur when the type of an expression is known before that expression has been typed, e.g. because the expression is argument to a function call. They can influence typing of that expression through what is called \tref{top-down inference}{type-system-top-down-inference}.}

A good example are arrays of mixed types. As mentioned in \Fullref{types-dynamic}, the compiler refuses \expr{[1, "foo"]} because it cannot determine an element type. Employing top-down inference, this can be overcome:

\haxe{assets/TopDownInference.hx}

Here, the compiler knows while typing \expr{[1, "foo"]} that the expected type is \type{Array<Dynamic>}, so the element type is \type{Dynamic}. Instead of the usual unification behavior where the compiler would attempt (and fail) to determine a \tref{common base type}{type-system-unification-common-base-type}, the individual elements are typed against and unified with \type{Dynamic}.

We have seen another interesting use of top-down inference when \tref{construction of generic type parameters}{type-system-generic-type-parameter-construction} was introduced:

\haxe{assets/GenericTypeParameter.hx}

The explicit types \type{String} and \type{haxe.Template} are used here to determine the return type of \expr{make}. This works because the method is invoked as \expr{make()}, so we know the return type will be assigned to the variables. Utilizing this information, it is possible to bind the unknown type \type{T} to \type{String} and \type{haxe.Template} respectively.

% this is not really top down inference
%Top-down inference is also utilized when dealing with \tref{enum constructors}{types-enum-constructor}:

%\haxe{assets/TopDownInference2.hx}

%The constructors \expr{TObject} and \expr{TFunction} of type \expr{ValueType} are recognized even though their containing module \type{Type} is not \tref{imported}{Import}. This is possible because the return type of \expr{Type.typeof("foo")} is known to be \expr{ValueType}.


\subsection{Limitations}
\label{type-system-inference-limitations}

Type inference saves a lot of manual type hints when working with local variables, but sometimes the type system still needs some help. In fact, it does not even try to infer the type of a \tref{variable}{class-field-variable} or \tref{property}{class-field-property} field unless it has a direct initialization.

There are also some cases involving recursion where type inference has limitations. If a function calls itself recursively while its type is not (completely) known yet, type inference may infer a wrong, too specialized type.

A different kind of limitation involves the readability of code. If type inference is overused it might be difficult to understand parts of a program due to the lack of visible types. This is particularly true for method signatures. It is recommended to find a good balance between type inference and explicit type hints.


\section{Modules and Paths}
\label{type-system-modules-and-paths}

\define{Module}{define-module}{All Haxe code is organized in modules, which are addressed using paths. In essence, each .hx file represents a module which may contain several types. A type may be \expr{private}, in which case only its containing module can access it.}

The distinction of a module and its containing type of the same name is blurry by design. In fact, addressing \expr{haxe.ds.StringMap<Int>} can be considered shorthand for \expr{haxe.ds.StringMap.StringMap<Int>}. The latter version consists of four parts:

\begin{enumerate}
	\item the package \expr{haxe.ds}
	\item the module name \expr{StringMap}
	\item the type name \type{StringMap}
	\item the type parameter \type{Int}
\end{enumerate}
If the module and type name are equal, the duplicate can be removed, leading to the \expr{haxe.ds.StringMap<Int>} short version. However, knowing about the extended version helps with understanding how \tref{module sub-types}{type-system-module-sub-types} are addressed.

Paths can be shortened further by using an \tref{import}{type-system-import}, which typically allows omitting the package part of a path. This may lead to usage of unqualified identifiers, for which understanding the \tref{resolution order}{type-system-resolution-order} is required.

\define{Type path}{define-type-path}{The (dot-)path to a type consists of the package, the module name and the type name. Its general form is \expr{pack1.pack2.packN.ModuleName.TypeName}.} 


\subsection{Module Sub-Types}
\label{type-system-module-sub-types}

A module sub-type is a type declared in a module with a different name than that module. This allows a single .hx file to contain multiple types, which can be accessed unqualified from within the module, and by using \expr{package.Module.Type} from other modules:

\begin{lstlisting}
var e:haxe.macro.Expr.ExprDef;
\end{lstlisting}

Here the sub-type \type{ExprDef} within module \expr{haxe.macro.Expr} is accessed. 

The sub-type relation is not reflected at run-time. That is, public sub-types become a member of their containing package, which could lead to conflicts if two modules within the same package tried to define the same sub-type. Naturally, the Haxe compiler detects these cases and reports them accordingly. In the example above \type{ExprDef} is generated as \type{haxe.macro.ExprDef}.

Sub-types can also be made private:

\begin{lstlisting}
private class C { ... }
private enum E { ... }
private typedef T { ... }
private abstract A { ... }
\end{lstlisting}

\define{Private type}{define-private-type}{A type can be made private by using the \expr{private} modifier. As a result, the type can only be directly accessed from within the \tref{module}{define-module} it is defined in.

Private types, unlike public ones, do not become a member of their containing package.}

The accessibility of types can be controlled more fine-grained by using \tref{access control}{lf-access-control}.



\subsection{Import}
\label{type-system-import}

If a type path is used multiple times in a .hx file, it might make sense to use an \expr{import} to shorten it. This allows omitting the package when using the type:

\haxe{assets/Import.hx}

With \expr{haxe.ds.StringMap} being imported in the first line, the compiler is able to resolve the unqualified identifier \expr{StringMap} in the \expr{main} function to this package. The module \type{StringMap} is said to be \emph{imported} into the current file.

In this example, we are actually importing a \emph{module}, not just a specific type within that module. This means that all types defined within the imported module are available:

\haxe{assets/Import2.hx}

The type \type{Binop} is an \tref{enum}{types-enum-instance} declared in the module \type{haxe.macro.Expr}, and thus available after the import of said module. If we were to import only a specific type of that module, e.g. \expr{import haxe.macro.Expr.ExprDef}, the program would fail to compile with \expr{Class not found : Binop}.

There are several aspects worth knowing about importing:

\begin{itemize}
	\item The bottommost import takes priority (detailed in \Fullref{type-system-resolution-order}).
	\item The \tref{static extension}{lf-static-extension} keyword \expr{using} implies the effect of \expr{import}.
	\item If an enum is imported (directly or as part of a module import), all its \tref{enum constructors}{types-enum-constructor} are also imported (this is what allows the \expr{OpAdd} usage in the above example).
\end{itemize}

Furthermore, it is also possible to import \tref{static fields}{class-field} of a class and use them unqualified:

\haxe{assets/Import3.hx}

Special care has to be taken with field names or local variable names that conflict with a package name: Since they take priority over packages, a local variable named \expr{haxe} blocks off usage the entire \expr{haxe} package.

\paragraph{Wildcard import}

Haxe allows using \expr{.*} to allow import of all modules in a package, all types in a module or all static fields in a type. It is important to understand that this kind of import only crosses a single level as we can see in the following example:

\haxe{assets/ImportWildcard.hx}

Using the wildcard import on \expr{haxe.macro} allows accessing \type{Expr} which is a module in this package, but it does not allow accessing \type{ExprDef} which is a sub-type of the \type{Expr} module. This rule extends to static fields when a module is imported.

When using wildcard imports on a package the compiler does not eagerly process all modules in that package. This means that these modules are never actually seen by the compiler unless used explicitly and are then not part of the generated output.

\paragraph{Import with alias}

If a type or static field is used a lot in an importing module it might help to alias it to a shorter name. This can also be used to disambiguate conflicting names by giving them a unique identifier.

\haxe{assets/ImportAlias.hx}

Here we import \expr{String.fromCharCode} as \expr{f} which allows us to use \expr{f(65)} and \expr{f(66)}. While the same could be achieved with a local variable, this method is compile-time exclusive and guaranteed to have no run-time overhead.

\since{3.2.0}

Haxe also allows the more natural \expr{as} in place of \expr{in}.


\subsection{Resolution Order}
\label{type-system-resolution-order}

Resolution order comes into play as soon as unqualified identifiers are involved. These are \tref{expressions}{expression} in the form of \expr{foo()}, \expr{foo = 1} and \expr{foo.field}. The last one in particular includes module paths such as \expr{haxe.ds.StringMap}, where \expr{haxe} is an unqualified identifier.  

We describe the resolution order algorithm here, which depends on the following state:

\begin{itemize}
	\item the declared \tref{local variables}{expression-var} (including function arguments)
	\item the \tref{imported}{type-system-import} modules, types and statics
	\item the available \tref{static extensions}{lf-static-extension}
	\item the kind (static or member) of the current field
	\item the declared member fields on the current class and its parent classes
	\item the declared static fields on the current class
	\item the \tref{expected type}{define-expected-type}
	\item the expression being \expr{untyped} or not
\end{itemize}

\todo{proper label and caption + code/identifier styling for diagram}

\input{assets/tikz/resolution-order.tex}

Given an identifier \expr{i}, the algorithm is as follows:

\begin{enumerate}
	\item If i is \expr{true}, \expr{false}, \expr{this}, \expr{super} or \expr{null}, resolve to the matching constant and halt.
	\item If a local variable named \expr{i} is accessible, resolve to it and halt.
	\item If the current field is static, go to \ref{resolution:static-lookup}.
	\item If the current class or any of its parent classes has a field named \expr{i}, resolve to it and halt.
	\item\label{resolution:static-extension} If a static extension with a first argument of the type of the current class is available, resolve to it and halt.
	\item\label{resolution:static-lookup} If the current class has a static field named \expr{i}, resolve to it and halt.
	\item\label{resolution:enum-ctor} If an enum constructor named \expr{i} is declared on an imported enum, resolve to it and halt.
	\item If a static named \expr{i} is explicitly imported, resolve to it and halt.
	\item If \expr{i} starts with a lower-case character, go to \ref{resolution:untyped}.
	\item\label{resolution:type} If a type named \expr{i} is available, resolve to it and halt.
	\item\label{resolution:untyped} If the expression is not in untyped mode, go to \ref{resolution:failure}
	\item If \expr{i} equals \expr{__this__}, resolve to the \expr{this} constant and halt.
	\item Generate a local variable named \expr{i}, resolve to it and halt.
	\item\label{resolution:failure} Fail
\end{enumerate}

For step \ref{resolution:type}, it is also necessary to define the resolution order of types:

\begin{enumerate}
	\item\label{resolution:import} If a type named \expr{i} is imported (directly or as part of a module), resolve to it and halt.
	\item If the current package contains a module named \expr{i} with a type named \expr{i}, resolve to it and halt.
	\item If a type named \expr{i} is available at top-level, resolve to it and halt.
	\item Fail
\end{enumerate}

For step \ref{resolution:import} of this algorithm as well as steps \ref{resolution:static-extension} and \ref{resolution:enum-ctor} of the previous one, the order of import resolution is important:

\begin{itemize}
	\item Imported modules and static extensions are checked from bottom to top with the first match being picked.
	\item Within a given module, types are checked from top to bottom.
	\item For imports, a match is made if the name equals.
	\item For \tref{static extensions}{lf-static-extension}, a match is made if the name equals and the first argument \tref{unifies}{type-system-unification}. Within a given type being used as static extension, the fields are checked from top to bottom.
\end{itemize}
