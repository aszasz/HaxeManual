\chapter{Tipos}% upstream: commit 79adb3f36faa851206ab38c055e80645cfd73628 on Oct 15, 2016 
\label{types}%ideally we try to keep the line number in sync with the original file to make reference easier. This is (at this point) a pure translation not a version

O Compilador de Haxe emprega um detalhado sistema de tipos que ajuda a detectar erros de tipagem em um programa durante a compilação. Um erro de tipagem é uma operação inválida sobre um dado tipo como a divisão por um String, a tentativa de acesso a um campo de um Integer ou a chamada a uma função com menos (ou mais) argumentos que o necessário.

Em algumas linguagens essa segurança adicional é custosa porque os programadores são forçados a indicar explicitamente os tipos em suas construções sintáticas:

\begin{lstlisting}
var myButton:MySpecialButton = new MySpecialButton(); // As3
MySpecialButton* myButton = new MySpecialButton(); // C++ 
\end{lstlisting}
As indicações explicitas de tipos não são requeridas em Haxe, porque o compilador pode \emph{inferir} o tipo:

\begin{lstlisting}
var myButton = new MySpecialButton(); // Haxe
\end{lstlisting}
Exploraremos a inferência de em detalhes depois em \Fullref{type-system-type-inference}. Por hora, é suficiente dizer que a variável \expr{myButton} no código acima é reconhecida como uma \emph{instância da classe} type{ MySpecialButton}.

O sistema de tipos de Haxe reconhece sete grupos de tipos\translatenote{a palavra type pode aparecer nessa tradução, ao invés de tipo, se facilitar a compreensão. Nesse caso texto dirá algo como ``o type convertido''}:

\begin{description}
 \item[\emph{Instância de classe}:] um objeto de uma dada classe ou interface
 \item[\emph{Instância de enumeração}:] um valor de uma enumeração de Haxe 
 \item[\emph{Estrutura}:] uma estrutura anônima, i.e., uma coleção de campos com nomes
 \item[\emph{Função}:] um tipo composto de vários vários argumentos e um retorno
 \item[\emph{Dinâmico}:] um tipo coringa que é compatível com qualquer tipo
 \item[\emph{Abstrato}:] um tipo no momento de compilação representado outro tipo diferente na execução
 \item[\emph{Monomorfo}:] um tipo desconhecido que deve mais tarde se tornar um tipo diferente
\end{description}

Nos próximos capítulos, cada um desses grupos de tipos será descrito, bem como as relações entre eles.

\define{Tipo composto}{define-compound-type}{Um tipo composto é um tipo que tem subtipos. Isso inclui qualquer tipo com \tref{parâmetros de tipo}{type-system-type-parameters} e as \tref{funções}{types-function}.}

% When generating the .md, the bellow line seems to generate the 
% Uncaught exception - 02-types.tex:85: characters 9086-9087: No match: TNewline
\section{Tipos básicos}
\label{types-basic-types}

Os \emph{tipos básicos} são \type{Bool}, \type{Float} and \type{Int}. Eles podem ser facilmente identificados na sintaxe por valores como


\begin{itemize}
	\item \expr{true} e \expr{false} para \type{Bool},
	\item \expr{1}, \expr{0}, \expr{-1} e \expr{0xFF0000} para \type{Int} e
	\item \expr{1.0}, \expr{0.0}, \expr{-1.0}, \expr{1e10} para \type{Float}.
\end{itemize}

Tipos básicos não são \tref{classes}{types-class-instance} em Haxe. Eles são implementados como \tref{tipos abstratos}{types-abstract} e estão amarrados ao ``gerenciamento de operadores'' interno ao compilador, como descrito nas seções seguintes.

\subsection{Tipos numéricos}
\label{types-numeric-types}

\define[Tipo]{Float}{define-float}{Representa um número em ponto flutuante IEEE com 64 bits e precisão dupla.}

\define[Tipo]{Int}{define-int}{Representa um número inteiro.}
Enquanto todo \type{Int} pode ser usado onde se espera um \type{Float} (\type{Int} \emph{é atribuível a} ou \emph{unifica com} \type{Float}), o contrário não é verdade: a atribuição de um \type{Float} a um \type{Int} pode tirar precisão e não é permitida implicitamente.

\subsection{Overflow}
\label{types-overflow}

Por razões de desempenho, o Compilador de Haxe não define um comportamento particular de overflow\translatenote{a condição a que se refere aqui como overflow é a codição que se atinge quando o resultado de uma operação é maior do que o espaço de memória destinado a guardá-lo, algumas publicações em português utilizam transbordamento e outras estouro, optamos  por manter o termo original overflow, como o fazem outras tantas publicações.}. A responsabilidade de verificar overflows recai sobre a plataforma do target. Aqui se vêem algumas observações especificas do comportamento de overflow:

\begin{description}
 \item[C++, Java, C\#, Neko, Flash:] Inteiros de 32 bits com sinal e comportamento usual de overflow
 \item[PHP, JS, Flash 8:] Não possuem tipo \emph{Int} nativo e haverá perda de precisão quando eles atingirem seu limite de ponto flutuante (2\textsuperscript{52})
\end{description}

Alternativamente, as classes \emph{haxe.Int32} e \emph{haxe.Int64} podem ser usadas para assegurar correto comportamento de overflow em qualquer plataforma, ao custo de eventuais cálculos adicionais.

\subsection{Operadores numéricos}
\label{types-numeric-operators}

\todo{make sure the types are right for inc, dec, negate, and bitwise negate}
\todo{While introducing the different operations, we should include that information as well, including how they differ with the "C" standard, see http://haxe.org/manual/operators}
Esta é a lista de operadores numéricos em Haxe, agrupados por prioridade decrescente.

\begin{center}
\begin{tabular}{| l | l | l | l | l |}
	\hline
	\multicolumn{5}{|c|}{Aritméticos} \\ \hline
	Operador & Operação & Argumento 1 & Argumento 2 & Retorno \\ \hline
	\expr{++}& incremento & \type{Int} & N/A & \type{Int}\\
	& & \type{Float} & N/A & \type{Float}\\
	\expr{--} & decremento & \type{Int} & N/A & \type{Int}\\
	& & \type{Float} & N/A & \type{Float}\\
	\expr{+} & adição & \type{Float} & \type{Float} & \type{Float} \\
	& & \type{Float} & \type{Int} & \type{Float} \\
	& & \type{Int} & \type{Float} & \type{Float} \\
	& & \type{Int} & \type{Int} & \type{Int} \\
	\expr{-} & subtração & \type{Float} & \type{Float} & \type{Float} \\
	& & \type{Float} & \type{Int} & \type{Float} \\
	& & \type{Int} & \type{Float} & \type{Float} \\
	& & \type{Int} & \type{Int} & \type{Int} \\
	\expr{*} & multiplicação & \type{Float} & \type{Float} & \type{Float} \\
	& & \type{Float} & \type{Int} & \type{Float} \\
	& & \type{Int} & \type{Float} & \type{Float} \\
	& & \type{Int} & \type{Int} & \type{Int} \\	
	\expr{/} & divisão & \type{Float} & \type{Float} & \type{Float} \\
	& & \type{Float} & \type{Int} & \type{Float} \\
	& & \type{Int} & \type{Float} & \type{Float} \\
	& & \type{Int} & \type{Int} & \type{Float} \\
	\expr{\%} & módulo & \type{Float} & \type{Float} & \type{Float} \\
	& & \type{Float} & \type{Int} & \type{Float} \\
	& & \type{Int} & \type{Float} & \type{Float} \\
	& & \type{Int} & \type{Int} & \type{Int} \\	 \hline
	\multicolumn{5}{|c|}{Comparação} \\ \hline
	Operador & Operação & Argumento 1 & Argumento 2 & Retorno \\ \hline
	\expr{==} & igual & \type{Float/Int} & \type{Float/Int} & \type{Bool} \\
	\expr{!=} & diferente & \type{Float/Int} & \type{Float/Int} & \type{Bool} \\
	\expr{<} & menor que & \type{Float/Int} & \type{Float/Int} & \type{Bool} \\
	\expr{<=} & menor ou igual a & \type{Float/Int} & \type{Float/Int} & \type{Bool} \\
	\expr{>} & maior que & \type{Float/Int} & \type{Float/Int} & \type{Bool} \\
	\expr{>=} & maior ou igual a & \type{Float/Int} & \type{Float/Int} & \type{Bool} \\ \hline
	\multicolumn{5}{|c|}{Bit-a-bit} \\ \hline
	Operador & Operação & Argumento 1 & Argumento 2 & Retorno \\ \hline
	\expr{\textasciitilde} & negação bit-a-bit & \type{Int} & N/A & \type{Int} \\	
	\expr{\&} & e bit-a-bit & \type{Int} & \type{Int} & \type{Int} \\	
	\expr{|} & ou bit-a-bit & \type{Int} & \type{Int} & \type{Int} \\	
	\expr{\^} & xor bit-a-bit & \type{Int} & \type{Int} & \type{Int} \\	
	\expr{<{}<} & shift à esquerda & \type{Int} & \type{Int} & \type{Int} \\  % FIXME traduzir shift?
	\expr{>{}>} & shift à direita & \type{Int} & \type{Int} & \type{Int} \\  % FIXME disable ligatures for \expr and other tt typesetting
	\expr{>{}>{}>} & shift à direita sem sinal & \type{Int} & \type{Int} & \type{Int} \\ \hline
\end{tabular}
\end{center}

\paragraph{Igualdade}

\emph{Para enumerações(Enums):}
\begin{description}
	\item[Enumerações sem parâmetros] sempre representam os mesmos valores, então \expr{MyEnum.A == MyEnum.A}.
    \item[Enumerações com parâmetros] podem ser comparadas com \expr{a.equals(b)} (que é um forma curta para \expr{Type.enumEquals()}).
\end{description}

\emph{Dinâmicos(Tipo Dynamic)}
Comparações envolvendo ao menos um valor Dynamic não são especificadas e "específicas a plataforma"

\subsection{Bool}
\label{types-bool}

\define[Type]{Bool(Booleano)}{define-bool}{Representa um valor que pode ser ou \emph{true} ou \emph{false} (verdadeiro ou falso)}

Valores do tipo \type{Bool} são uma ocorrência comum em \emph{condicionais} como \tref{\expr{if}}{expression-if} e \tref{\expr{while}}{expression-while}. Os \emph{operadores} seguintes aceitam e retornam valores \type{Bool}:

\begin{itemize}
    \item \expr{\&\&} (e)
	\item \expr{||} (ou)
	\item \expr{!} (não)
\end{itemize}

O Haxe garante que as expressões booleanas compostas são avaliadas da esquerda para a direita e apenas até onde for necessário em tempo de execução. Por exemplo, uma expressão como \expr{A \&\& B}resolverá primeiro \expr{A}, e resolverá \expr{B} apenas se \expr{A} for verdadeira. Da mesma forma, expressões como \expr{A || B} não resolverão B se \expr{A} for verdadeira. Isso é importante em casos como:

\begin{lstlisting}
if (object != null && object.field == 1) { }
\end{lstlisting}

Acessar \expr{object.field} se \expr{object} for \expr{null} levaria a um erro em tempo de execução, mas a verificação de \expr{object!=null} previne isso.




\subsection{Void}
\label{types-void}

\define[Type]{Void}{define-void}{Void indica a ausência de um tipo. É usado para expressar que algo (normalmente uma função) não tem um valor}

\type{Void} é um caso especial no sitema de tipos porque não é em si um tipo. Ele é usado para expressar a ausência de um tipo. Já vimos Void no exemplo inicial ``Hello World'':
\todo{please review, doubled content}

\haxe{assets/HelloWorld.hx}

O tipo função será explorado em detalhe na seção \Fullref{types-function}, mas uma rápida visão é útil aqui: O tipo da função \expr{main} no exemplo anterior é \type{Void-->Void}, que se lê: ``não tem argumentos e nem valor de retorno''.
Haxe não permite campos e variáveis do tipo \type{Void} e reclamará se uma tentativa de os declarar assim for feita:
\todo{review please, sounds weird}

\begin{lstlisting}
// Arguments and variables of type Void
// are not allowed
var x:Void;
\end{lstlisting}



\section{Nulabilidade}
\label{types-nullability}

\define{nullable}{Um tipo em Haxe é considerado \emph{nullable}\translatenote{a pronúncia é nu-la-bol, a tradução seria ``nulável'', mas usaremos o termo em inglês}se \expr{null} é um valor válido para ele}

É comum que linguagens de programação tenham uma única definição clara de nulabilidade. No entanto, o Haxe tem que se comprometer em relação a esse assunto devido a natureza das linguagens dos targets: Enquanto algumas delas permitem e, de fato, padronizam \expr{null} para tudo, outras nem mesmo permitem \expr{null} para certos tipos. Isso gera a necessidade de criar a distinção entre dois tipos de targets:

\define{Target estático}{define-static-target}{Targets estáticos empregam sistemas de tipificação onde \expr{null} é um valor inválido para os tipos básicos. Esse é o caso de \target{Flash}, \target{C++}, \target{Java} and \target{C\#}.}

\define{Target dinâmico}{define-dynamic-target}{Dynamic targets are more lenient with their types and allow \expr{null} values for basic types. This applies to the \target{JavaScript}, \target{PHP}, \target{Neko} and \target{Flash 6-8} targets.}

Não há nada para se preocupar quando se trabalha com \expr{null} sobre targets dinâmicos; no entanto targets estáticos demandam alguma atenção. Para começar, tipos básicos são inicializados com seus valores padrão.
\todo{for starters...please review}

\define{Valores default}{define-default-value}{
	Tipos básicos tem os seguintes valores default em targets estáticos:
	\begin{description}
		\item[\type{Int}:] \expr{0}
		\item[\type{Float}:] \expr{NaN} on \target{Flash}, \expr{0.0} nos outros targets estáticos
		\item[\type{Bool}:] \expr{false}
	\end{description}
}

Como consequência o Compilador de Haxe não permite que se atribua \expr{null} a um tipo básico em targets estáticos. Para conseguir isso, o tipo básico tem que ser envelopado como \type{Null$<$T$>$}:

\begin{lstlisting}
// error em plataformas estáticas
var a:Int = null;
var b:Null<Int> = null; // permitido
\end{lstlisting}

Similarmente, tipos básicos não podem ser comparados a \expr{null}, a não ser que sejam envelopados:

\begin{lstlisting}
var a : Int = 0;
// error on static platforms
if( a == null ) { ... }
var b : Null<Int> = 0;
if( b != null ) { ... } // allowed
\end{lstlisting}

Essa restrição se estende a todas as situações onde acontece a \tref{unificação}{type-system-unification} 

\define[Type]{\expr{Null<T>}}{define-null-t}{Em targets estáticos os tipos \type{Null<Int>}, \type{Null<Float>} and \type{Null<Bool>} podem ser utilizados para permitir \expr{null} como um valor. Em targets dinâmicos esse envelopamento não tem nenhum efeito. \type{Null<T>} também pode ser usados com outros tipos de forma a documentar que null é um valor permitido.}

Ser um valor \expr{null}  é ``escondido'' em \type{Null$<$T$>$} ou \type{Dynamic} e atribuído a um tipo básico. o valor default da plataforma target é utilizado.

\begin{lstlisting}
var n : Null<Int> = null;
var a : Int = n;
trace(a); // 0 em plataformas estáticas 
\end{lstlisting}



\subsection{Argumentos Opcionais e Nulabilidade}
\label{types-nullability-optional-arguments}

Argumentos opcionais também precisam ser levados em conta quando se considera a nulabilidade.

Em especial, deva haver uma disitinção entre argumentos opcionais \emph{nativos} que não são nuláveis e argumentos opcionais específicos do Haxe que podem ser nuláveis. A distinção pode ser feita utilizando o argumento opcional ponto-de-interrogação:

\begin{lstlisting}
// x is a native Int (not nullable)
function foo(x : Int = 0) {...}
// y is Null<Int> (nullable)
function bar( ?y : Int) {...}
// z is also Null<Int>
function opt( ?z : Int = -1) {...}
\end{lstlisting}
\todo{Is there a difference between \type{?y : Int} and \type{y : Null$<$Int$>$} or can you even do the latter? Some more explanation and examples with native optional and Haxe optional arguments and how they relate to nullability would be nice.}

\trivia{Argumentos x Parametros}{Em algumas outras linguagens de programação, \emph{argumentos} e \emph{parâmetros} são termos intercambiáveis. Em Haxe, \emph{argumento} é usado quando nos referimos a métodos e \emph{parâmetro} quando nos referimos a \Fullref{type-system-type-parameters}.}

\section{Instância de Classe}
\label{types-class-instance}

Similar a muitas linguagens orientadas a objeto, classes são a estrutura de dados primária para a maioria dos programas em Haxe. Cada classe de Haxe tem um nome explícito, um caminho específico e zero ou mais campos. Aqui nos focaremos na estrutura geral de classes e em suas relações e deixaremos os detalhes dos campos da classe para \Fullref{class-field}.
\todo{please review future tense}

O seguinte exemplo de código serve de base para o resto dessa seção:

\haxe{assets/Point.hx}

Semanticamente, essa classe representa um ponto discreto em um espaço bidimensional - mas isso não é importante aqui. Vamos, ao invés disso, descrever a estrutura: 

\begin{itemize}
	\item A palavra-chave \expr{class} informa que estamos declarando uma classe.
	\item \type{Point} é o nome da classe e poderia ser qualquer nome de acordo com \tref{regras de identificadores de tipos}{define-identifier}.
	\item Entre chaves \expr{$\left\{\right\}$} estão os campos da classe,
	\item que consistem de dois campos \emph{variáveis} \expr{x} e \expr{y} do tipo \type{Int},
	\item seguidos por um campo de \emph{function} especial, chamada \expr{new}, que é o \emph{constructor} da classe,
	\item bem como uma função normal \expr{toString}.
\end{itemize}
Há um tipo especial em Haxe que é compatível com todas as classes:

\define[Type]{\expr{Class$<$T$>$}}{define-class-t}{Esse tipo é compatível com todos os tipos de classe, o que siginifica que todas as classes (não suas instâncias) podem ser atribuídas a ele. Durante a compilação, \type{Class<T>} é o tipo base comum de todos os tipos de classes. Entretanto, essa relação não é refletida no código gerado.

Esse tipo é útil quando uma API exige que um valor seja \emph{uma} classe, mas nenhuma específica. Isso se aplica a diversos métodos da \tref{Haxe reflection API}{std-reflection}.}

\subsection{Constructor de classe}
\label{types-class-constructor}

Instâncias de classes são criadas pela chamada do constructor de classe - um processo comumente referido por \emph{instanciação}. Outro nome para instâncias de classes é \emph{objeto}. Apesar disso, preferimos o termo instância de classe, por enfatizar a analogia entre classes/instâncias de classes e \tref{enums/instâncias de enums}{types-enum-instance}.

\begin{lstlisting}
var p = new Point(-1, 65);
\end{lstlisting}
Isso proporcionará uma instância da classe i\type{Point}, que foi alocada em uma variável de nome p. O constructor de \type{Point} recebe os dois argumentos \expr{-1} e \expr{65} e os atribui as variáveis de instância \expr{x} e \expr{y} respectivamente (compare sua definição em \Fullref{types-class-instance}). Revisitaremos o significado exato da expressão \expr{new} mais tarde em \ref{expression-new}. Por ora, apenas pensamos nele como o constructor de classe e retornando o objeto apropriado.



\subsection{Herança}
\label{types-class-inheritance}

Classes podem herdar de outras classes, o que em Haxe é indicado pela palavra-chave \expr{extends}:

\haxe{assets/Point3.hx}
Essa relação é comumente descrita como "é um" (\type{Point3} é um \type{Point}): Qualquer instância da classe \type{Point3} é também uma instância da classe \type{Point}. \type{Point} é então chamada de \emp
h{classe mãe} de \type{Point3}, que é a  \emph{classe filha} de \type{Point}. Uma classe pode ter muitas classes filhas, mas unicamente uma classe mãe. O termo ``uma classe mãe da classe X'' normalmente se rerfere à classe mãe direta, o classe mãe da classe mãe por aí vai. \translatornote{confusco como no original}

O código acima é bastante similar ao código original da classe \type{Point}, com dois novos constructs envolvidos:
\begin{itemize}
 \item \expr{extends Point} indica que essa classe herda de classe \type{Point}
 \item \expr{super (x,y)} é a chamada ao constructror da classe mãe, nesse caso \expr{Point.new}
\end{itemize}
Não é necessário para as classes filhas definirem seus próprios constructors, mas se o fizerem, uma chamada a \expr{super()} é obrigatória. Não como outras linguagens orientadas a objeto, essa chamada pode aparecer em qualquer ponto do código do constructor e não tem que ser a primeira expressão.

Uma classe pode sobreescrever \tref{metódos}{class-field-method} de sua classe mãe, o que exige explicitamente a palavra-chave \expr{override}. Os efeitos e restrições do disso são detalhados adiante em  \Fullref{class-field-overriding}.

\subsection{Interfaces}
\label{types-interfaces}

Uma interface pode ser entendida como a assinatura de uma classe porque ela descreve os campos públicos de uma classe. Interfaces não fornecem implementações mas informações estruturais puras:

\begin{lstlisting}
interface Printable {
	public function toString():String;
}
\end{lstlisting}
A sintaxe é similar a de classes, com as seguintes exceções:

\begin{itemize}
	\item a palavra-chave \expr{interface} é usada ao invés da palavra-chave \expr{class} 
	\item funções não tem nenhuma \tref{expressions}{expression}
	\item todo campo deve ter um tipo explícito 
\end{itemize}
Interfaces, não como \tref{structural subtyping}{type-system-structural-subtyping}, descrevem uma \emph{relação estática} entre classes. Uma dada class é considerada como compatível a uma interface unicamente se ela o declarar explicitamente:

\begin{lstlisting}
class Point implements Printable { }
\end{lstlisting}
Aqui, a palavra-chave \expr{implements} indica que \type{Point} tem uma relação "é uma" com \type{Printable}, i.e. cada instância de \type{Point} é também uma instância de \type{Printable}. Enquanto uma classe só pode ter uma classe mãe, ela pode implementar múltiplas interfaces através de múltiplas palavras-chave \expr{implements}:

\begin{lstlisting}
class Point implements Printable
  implements Serializable
\end{lstlisting}

O compilador verifica se a premissa \expr{implements} é válida. Isso é, ele garante que a classe  realmente implementa todos os campos exigidos pela interface. Um campo é considerado implementado se a classe ou alguma de suas classes ancestrais oferece uma implementação.

Campos de interfaces não são limitados a métodos. Eles podem ser variáveis e propriedades também:

\haxe{assets/InterfaceWithVariables.hx}

Interfaces podem estender múltiplas outras interfaces usando a palavra-chave \expr{extends}:
\begin{lstlisting}
interface Debuggable extends Printable extends Serializable
\end{lstlisting}


\trivia{A Sintaxe de Implements}{Versões de Haxe prévias a 3.0 exigiam que múltiplas palavras-chave \expr{implements} fossem separadas por uma vírgula. Nós decidimos aderir ao ``padrão de facto'' do Java e nos livramos da vírgula. Essa foi uma das mudanças de incompatibilidade entre Haxe 2 and 3.}


\section{Instância de Enum}
\label{types-enum-instance}

Haxe oferece poderosos tipos enumeradores (enumerações ou enum). que são na verdade \emph{tipos de dados algébricos} (Algebraic Data Type)\footnote{\url{http://en.wikipedia.org/wiki/Algebraic_data_type}}. Ainda que não possam ter nenhuma  \tref{expressão}{expression}, eles são muito úteis para a descrição de estruturas de dados:

\haxe{assets/Color.hx}
Semanticamente, esse enum descreve uma cor que é vermelha (red), verde(green), azul(blue) ou um valor especificado de RGB. A estrutura sintática é conforme segue:
\begin{itemize}
	\item A palavra-chave \expr{enum} indica que estamos declarando um enum.
	\item \type{Color} é o nome do enum, que poderia ser qualquer um que se conformasse com as regras de \tref{identificadores de tipo}{define-identifier}.
	\item Cercado por chaves \expr{$\left\{\right\}$} estão os \emph{constructors de enums},
	\item que são \expr{Red}, \expr{Green} e \expr{Blue} sem argumentos,
	\item bem como \expr{Rgb} que toma três argumentos \type{Int} denominados \expr{r}, \expr{g} and \expr{b}.
\end{itemize}
O sistema de tipos de Haxe oferece um tipo que unifica com todos os tipos de enums:

\define[Type]{\expr{Enum$<$T$>$}}{define-enum-t}{Esse tipo é compatível com todos os tipos enum. Em tempo de compilação, \type{Enum<T>} pode ser visto como a base comum de tipos de todos os tipos enum. Entretanto, essa relação não é refletida no código gerado.} 
\todo{Same as in 2.2, what is \type{Enum$<$T$>$} syntax?}

\subsection{Enum Constructor}
\label{types-enum-constructor}

Similar a classes e a seus construtores, enums oferecem uma maneira de instanciamento pelo uso de um de seus constructors. Entretanto, diferente de classes, enums fornecem muitos constructors. que podem ser facilmente utilizados através de seus nomes:


\begin{lstlisting}
var a = Red;
var b = Green;
var c = Rgb(255, 255, 0);
\end{lstlisting}
Nesse código o tipo das variáveis \expr{a}, \expr{b} e \expr{c} é \type{Color}. A variável \expr{c} é inicializada usando o constructor \{Rgb} com argumentos.

Todas as instâncias de enum podem ser atribuídas a um tipo especial de nome \type{EnumValue}.

\define[Type]{EnumValue}{define-enum-value}{EnumValue é um tipo especial que unifica com todas as instâncias de enums. Ele é usado pela Biblioteca Padrão do Haxe para fornecer certas operações para todas as instâncias de enum e pode ser empregado no "código usuário" apropriadamente em casos onde a API exige \emph{uma} instância de enum, mas não uma específica.}

É importante distinguir tipos enum e constructors de enums, como esse exemplo demonstra:

\haxe{assets/EnumUnification.hx}

Se a linha comentada tiver o sinal de cometário removido, o programa não compila porque \expr{Red} (um constructor de enum) não pode ser atribuído a uma variável do tipo \type{Enum<Color>} (um tipo enum). A relação é análoga a relação de uma classe e sua instância.

\trivia{Parâmetro de tipo concreto para \type{Enun$<$T$>$}}{Um dos revisores deste manual estava confuso sobre a diferença entre \type{Color} e \type{Enum<Color>} no exemplo acima. De fato, o uso de um parâmetro de tipo concreto ali é desnecessário e tem apenas o propósito de demonstração. Geralmente nós omitiríamos o tipo ali e deixariamos a \tref{inferência de tipos}{type-system-type-inference} tratar disso.

Entretanto, o tipo inferido seria diferente de \type{Enum<Color>}. O compilador infere um pseudo-tipo que tem o enum constructor como um de seus ``campos''. Já no Haxe 3.2.0 não é possível expressar este tipo por sintaxe, mas também nunca é necessário fazê-lo.}



\subsection{Usando enums}
\labe{types-enum-using}

Enums são uma boa escolha se apenas um conjunto finito de valores deve ser permitido. Os \tref{constructors}{types-enum-constructor} individuais representam as variantes permitidas e habilitam o compilador a verificar se todos os valores possíveis são respeitados. Isso pode ser visot aqui:

\haxe{assets/Color2.hx}

Depois de receber o valor de \expr{color} pela atribuição do valor de retorno de \expr{getColor()} à ela, uma \tref{expressão \expr{switch}}{expression-switch} é usada para para direcionar o procedimento dependendo do valor. Os primeiros trẽs casos: \expr{Red}, \expr{Green} e \expr{Blue} são triviais e correspondem aos constructors de \type{Color} que não têm argumentos. O caso final \expr{Rgb(r, g, b)} mostra como os valores dos argumentos de um constructor podem ser extraídos: eles estão disponíveis como variáveis locais internas oa corpo da experssão \expr{case}, assim como se uma \tref{expressão \expr{var}}{expression-var} tivesse sido usada.

Informações avançadas sobre o uso da expressão \expr{switch} serão exploradas mais tarde na seção sobre \tref{pattern matching}{lf-pattern-matching}.


\section{Estruturas Anônimas}
\label{types-anonymous-structures}

Estruturas anônimas podem ser usadas para agrupar dados sem explicitamente criar um tipo. O exemplo seguinte cria uma estrutura com dois campos: \expr{x} e \expr{name}, e a inicializa seus valores para \expr{12} \expr{"foo"} respectivamente:

\haxe{assets/Structure.hx}
As regras gerais de sintaxe seguem:

\begin{enumerate}
    \item Uma estrutura é cercada por chaves \expr{$\left\{\right\)$ e 
    \item Contém uma lista \emph{separada por vírgulas} de pares-chave-valor.
    \item Um \emph{dois pontos} separa a chave, que deve ser um \tref{identificador}{define-identifier} válido, do valor.
    \item \label{valueanytype} O valor pode ser qualquer expressão Haxe.
\end{enumerate} 
Regra \ref{valueanytype} implica que estruturas podem ser aninhadas e complexas, e.g.:

\todo{please reformat}

\begin{lstlisting}
var user = {
  name : "Nicolas",
	age : 32,
	pos : [
	  { x : 0, y : 0 },
		{ x : 1, y : -1 }
  ],
};
\end{lstlisting}
Campos de estruturas, como de classes, são acessados usando um \emph{ponto} (\expr{.}) tal como:

\begin{lstlisting}
// get value of name, which is "Nicolas"
user.name;
// set value of age to 33
user.age = 33;
\end{lstlisting}
Vale a pena observar que o uso de estruturas anônimas não subverte o sistema de tipagem. O compilador assegura que unicamente campos disponíveis sejam acessados, o que significa que o seguinte programa não compila:

\begin{lstlisting}
class Test {
  static public function main() {
    var point = { x: 0.0, y: 12.0 };
    // { y : Float, x : Float } has no field z
    point.z;
  }
}
\end{lstlisting}
A mensagem de erro indica que o compilador conhece o tipo de \expr{point}: È uma estrutura com campos\expr{x} e \expr{y} do tipo \type {Float}. Já que não tem campo \expr{z}, o acesso falha.
O tipo de \expr{point} é conhdecido por \tref{inferência de tipos}{type-system-type-inference}, que gentilmente nos poupa do uso de tipos explícitos para variáveis locais. Entretanto, se \expr{point} fosse um campo. tipagem explícita seria necssária:

\begin{lstlisting}
class Path {
    var start : { x : Int, y : Int };
    var target : { x : Int, y : Int };
    var current : { x : Int, y : Int };
}
\end{lstlisting}
Para evitar esse tipo de declaração de tipo redundante, especialmente para estruturas mais complexas, é aconselhado o uso de um \tref{typedef}{type-system-typedef}:

\begin{lstlisting}
typedef Point = { x : Int, y : Int }

class Path {
    var start : Point;
    var target : Point;
    var current : Point;
}
\end{lstlisting}


\subsection{JSON para Valores de Estruturas}
\label{types-structure-json}

Também é possível usar \emph{JavaScript Object Notation} para estruturas através de \emph{string literals}  para os valores-chaves:

\begin{lstlisting}
var point = { "x" : 1, "y" : -5 };
\end{lstlisting}
Enquanto qualquer string literal é permitito, o campo é unicamente considerado parte do tipo se é um \tref{identificador Haxe}{define-identifier} válido. De outra forma, a sintaxe de Haxe não permite a expressão de acesso a tal campo, e \tref{reflexão}{std-reflection} tem que ser empregada através do uso de \expr{Reflect.field} e \expr{Reflect.setField}.

\subsection{Notação de Classe para Tipos Estrutura}
\label{types-structure-class-notation}

Quando define um tipo estrutura, Haxe permite o uso da mesma sintaxe descrita em \Fullref{class-field}. O seguinte \tref{typedef}{type-system-typedef} declara um tipo \type{Point} com campos de variáveis \expr{x} e \expr{y} do tipo \type{Int}:

\begin{lstlisting}
typedef Point = {
    var x : Int;
    var y : Int;
}
\end{lstlisting}

\subsection{Campos opcionais}
\label{types-structure-optional-fields}

\todo{I don't really know how these work yet.}

\subsection{Impact sobre o desempenho}
\label{types-structure-performance}

O uso de estrututas e. por extensão, \tref{subtipagem estrutural}{type-system-structural-subtyping} não tem impacto no desempenho na compilação para \tref{targets dinâmicos}{define-dynamic-target}. Entretanto, em \tref{targets estáticos}{define-static-target} uma busca dinâmica tem de ser executada, o que é tipicamente mais lento do que um acesso a um campo estático.




\section{Function Type}
\label{types-function}

\todo{It seems a bit convoluted explanations. Should we maybe start by "decoding" the meaning of  Void -> Void, then Int -> Bool -> Float, then maybe have samples using \$type}

The function type, along with the \tref{monomorph}{types-monomorph}, is a type which is usually well-hidden from Haxe users, yet present everywhere. We can make it surface by using \expr{\$type}, a special Haxe identifier which outputs the type its expression has during compilation :

\haxe{assets/FunctionType.hx}

There is a strong resemblance between the declaration of function \expr{test} and the output of the first \expr{\$type} expression, yet also a subtle difference:

\begin{itemize}
	\item \emph{Function arguments} are separated by the special arrow token \expr{->} instead of commas, and
	\item the \emph{function return type} appears at the end after another \expr{->}.
\end{itemize}

In either notation it is obvious that the function \expr{test} accepts a first argument of type \type{Int}, a second argument of type \type{String} and returns a value of type \type{Bool}. If a call to this function, such as \expr{test(1, "foo")}, is made within the second \expr{\$type} expression, the Haxe typer checks if \expr{1} can be assigned to \type{Int} and if \expr{"foo"} can be assigned to \type{String}. The type of the call is then equal to the type of the value \expr{test} returns, which is \type{Bool}.

If a function type has other function types as argument or return type, parentheses can be used to group them correctly. For example, \type{Int -> (Int -> Void) -> Void} represents a function which has a first argument of type \type{Int}, a second argument of function type \type{Int -> Void} and a return of \type{Void}.



\subsection{Optional Arguments}
\label{types-function-optional-arguments}

Optional arguments are declared by prefixing an argument identifier with a question mark \expr{?}:

\haxe[label=assets/OptionalArguments.hx]{assets/OptionalArguments.hx}
Function \expr{test} has two optional arguments: \expr{i} of type \type{Int} and \expr{s} of \type{String}. This is directly reflected in the function type output by line 3. 
This example program calls \expr{test} four times and prints its return value.

\begin{enumerate}
	\item The first call is made without any arguments.
	\item The second call is made with a singular argument \expr{1}.
	\item The third call is made with two arguments \expr{1} and \expr{"foo"}.
	\item The fourth call is made with a singular argument \expr{"foo"}.
\end{enumerate}
The output shows that optional arguments which are omitted from the call have a value of \expr{null}. This implies that the type of these arguments must admit \expr{null} as value, which raises the question of its \tref{nullability}{types-nullability}. The Haxe Compiler ensures that optional basic type arguments are nullable by inferring their type as \type{Null<T>} when compiling to a \tref{static target}{define-static-target}.

While the first three calls are intuitive, the fourth one might come as a surprise: It is indeed allowed to skip optional arguments if the supplied value is assignable to a later argument.


\subsection{Default values}
\label{types-function-default-values}

Haxe allows default values for arguments by assigning a \emph{constant value} to them:

\haxe{assets/DefaultValues.hx}
This example is very similar to the one from \Fullref{types-function-optional-arguments}, with the only difference being that the values \expr{12} and \expr{"bar"} are assigned to the function arguments \expr{i} and \expr{s} respectively. The effect is that the default values are used instead of \expr{null} should an argument be omitted from the call.

%TODO: Default values do not imply nullability, even if the value is \expr{null}. 

Default values in Haxe are not part of the type and are not replaced at call-site (unless the function is \tref{inlined}{class-field-inline}, which can be considered as a more typical approach. On some targets the compiler may still pass \expr{null} for omitted argument values and generate code similar to this into the function:
\begin{lstlisting}
	static function test(i = 12, s = "bar") {
		if (i == null) i = 12;
		if (s == null) s = "bar";
		return "i: " +i + ", s: " +s;
	}
\end{lstlisting}
This should be considered in performance-critical code where a solution without default values may sometimes be more viable.




\section{Dynamic}
\label{types-dynamic}

While Haxe has a static type system, this type system can, in effect, be turned off by using the \type{Dynamic} type. A \emph{dynamic value} can be assigned to anything; and anything can be assigned to it. This has several drawbacks:

\begin{itemize}
	\item The compiler can no longer type-check assignments, function calls and other constructs where specific types are expected.
	\item Certain optimizations, in particular when compiling to static targets, can no longer be employed.
	\item Some common errors, e.g. a typo in a field access, can not be caught at compile-time and likely cause an error at runtime.
	\item \Fullref{cr-dce} cannot detect used fields if they are used through \type{Dynamic}.
\end{itemize}
It is very easy to come up with examples where the usage of \type{Dynamic} can cause problems at runtime. Consider compiling the following two lines to a static target:

\begin{lstlisting}
var d:Dynamic = 1;
d.foo;
\end{lstlisting}

Trying to run a compiled program in the Flash Player yields an error \texttt{Property foo not found on Number and there is no default value}. Without \type{Dynamic}, this would have been detected at compile-time.

\trivia{Dynamic Inference before Haxe 3}{The Haxe 3 compiler never infers a type to \type{Dynamic}, so users must be explicit about it. Previous Haxe versions used to infer arrays of mixed types, e.g. \expr{[1, true, "foo"]}, as \type{Array<Dynamic>}. We found that this behavior introduced too many type problems and thus removed it for Haxe 3.}

Use of \type{Dynamic} should be minimized as there are better options in many situations but sometimes it is just practical to use it. Parts of the Haxe \Fullref{std-reflection} API use it and it is sometimes the best option when dealing with custom data structures that are not known at compile-time.

\type{Dynamic} behaves in a special way when being \tref{unified}{type-system-unification} with a \tref{monomorph}{types-monomorph}. Monomorphs are never bound to \type{Dynamic} which can have surprising results in examples such as this:

\haxe{assets/DynamicInferenceIssue.hx}

Although the return type of \expr{Json.parse} is \type{Dynamic}, the type of local variable \expr{json} is not bound to it and remains a monomorph. It is then inferred as an \tref{anonymous structure}{types-anonymous-structure} upon the \expr{json.length} field access, which causes the following \expr{json[0]} array access to fail. In order to avoid this, the variable \expr{json} can be explicitly typed as \type{Dynamic} by using \expr{var json:Dynamic}.

\trivia{Dynamic in the Standard Library}{Dynamic was quite frequent in the Haxe Standard Library before Haxe 3. With the continuous improvements of the Haxe type system the occurrences of Dynamic were reduced over the releases leading to Haxe 3.}

\subsection{Dynamic with Type Parameter}
\label{types-dynamic-with-type-parameter}

\type{Dynamic} is a special type because it allows explicit declaration with and without a \tref{type parameter}{type-system-type-parameters}. If such a type parameter is provided, the semantics described in \Fullref{types-dynamic} are constrained to all fields being compatible with the parameter type:

\begin{lstlisting}
var att : Dynamic<String> = xml.attributes;
// valid, value is a String
att.name = "Nicolas";
// dito (this documentation is quite old)
att.age = "26";
// error, value is not a String
att.income = 0;
\end{lstlisting}


\subsection{Implementing Dynamic}
\label{types-dynamic-implemented}

Classes can \tref{implement}{types-interfaces} \type{Dynamic} and \type{Dynamic$<$T$>$} which enables arbitrary field access. In the former case, fields can have any type, in the latter, they are constrained to be compatible with the parameter type:

\haxe{assets/ImplementsDynamic.hx}

Implementing \type{Dynamic} does not satisfy the requirements of other implemented interfaces. The expected fields still have to be implemented explicitly.

Classes that implement \type{Dynamic} (with or without type parameter) can also utilize a special method named \expr{resolve}. If a \tref{read access}{define-read-access} is made and the field in question does not exist, the \expr{resolve} method is called with the field name as argument:

\haxe{assets/DynamicResolve.hx}



\section{Abstract}
\label{types-abstract}

An abstract type is a type which is actually a different type at run-time. It is a compile-time feature which defines types ``over'' concrete types in order to modify or augment their behavior:

\haxe[firstline=1,lastline=5]{assets/MyAbstract.hx}
We can derive the following from this example:

\begin{itemize}
	\item The keyword \expr{abstract} denotes that we are declaring an abstract type.
	\item \type{AbstractInt} is the name of the abstract and could be anything conforming to the rules for type identifiers.
	\item Enclosed in parenthesis \expr{()} is the \emph{underlying type} \type{Int}.
	\item Enclosed in curly braces \expr{$\left\{\right\}$} are the fields,
	\item which are a constructor function \expr{new} accepting one argument \expr{i} of type \type{Int}.
\end{itemize}

\define{Underlying Type}{define-underlying-type}{The underlying type of an abstract is the type which is used to represent said abstract at runtime. It is usually a concrete (i.e. non-abstract) type but could be another abstract type as well.}

The syntax is reminiscent of classes and the semantics are indeed similar. In fact, everything in the ``body'' of an abstract (that is everything after the opening curly brace) is parsed as class fields. Abstracts may have \tref{method}{class-field-method} fields and non-\tref{physical}{define-physical-field} \tref{property}{class-field-property} fields.

Furthermore, abstracts can be instantiated and used just like classes:

\haxe[firstline=7,lastline=12]{assets/MyAbstract.hx}
As mentioned before, abstracts are a compile-time feature, so it is interesting to see what the above actually generates. A suitable target for this is \target{JavaScript}, which tends to generate concise and clean code. Compiling the above (using \texttt{haxe -main MyAbstract -js myabstract.js}) shows this \target{JavaScript} code:

\begin{lstlisting}
var a = 12;
console.log(a);
\end{lstlisting}
The abstract type \type{Abstract} completely disappeared from the output and all that is left is a value of its underlying type, \type{Int}. This is because the constructor of \type{Abstract} is inlined - something we shall learn about later in the section \Fullref{class-field-inline} - and its inlined expression assigns a value to \expr{this}. This might be surprising when thinking in terms of classes. However, it is precisely what we want to express in the context of abstracts. Any \emph{inlined member method} of an abstract can assign to \expr{this}, and thus modify the ``internal value''.


A good question at this point is ``What happens if a member function is not declared inline'' because the code obviously has to go somewhere. Haxe creates a private class, known to be the \emph{implementation class}, which has all the abstract member functions as static functions accepting an additional first argument \expr{this} of the underlying type. While technically this is an implementation detail, it can be used for \tref{selective functions}{types-abstract-selective-functions}.



\trivia{Basic Types and abstracts}{Before the advent of abstract types, all basic types were implemented as extern classes or enums. While this nicely took care of some aspects such as \type{Int} being a ``child class'' of \type{Float}, it caused issues elsewhere. For instance, with \type{Float} being an extern class, it would unify with the empty structure \expr{\{\}}, making it impossible to constrain a type to accepting only real objects.}




\subsection{Implicit Casts}
\label{types-abstract-implicit-casts}

Unlike classes, abstracts allow defining implicit casts. There are two kinds of implicit casts:

\begin{description}
	\item[Direct:] Allows direct casting of the abstract type to or from another type. This is defined by adding \expr{to} and \expr{from} rules to the abstract type and is only allowed for types which unify with the underlying type of the abstract.
	\item[Class field:] Allows casting via calls to special cast functions. These functions are defined using \expr{@:to} and \expr{@:from} metadata. This kind of cast is allowed for all types.
\end{description}
The following code example shows an example of \emph{direct} casting:

\haxe{assets/ImplicitCastDirect.hx}
We declare \type{MyAbstract} as being \expr{from Int} and \expr{to Int}, meaning it can be assigned from \type{Int} and assigned to \type{Int}. This is shown in lines 9 and 10, where we first assign the \type{Int} \expr{12} to variable \expr{a} of type \type{MyAbstract} (this works due to the \expr{from Int} declaration) and then that abstract back to variable \expr{b} of type \type{Int} (this works due to the \expr{to Int} declaration).

Class field casts have the same semantics, but are defined completely differently:

\haxe{assets/ImplicitCastField.hx}
By adding \expr{@:from} to a static function, that function qualifies as implicit cast function from its argument type to the abstract. These functions must return a value of the abstract type. They must also be declared \expr{static}.

Similarly, adding \expr{@:to} to a function qualifies it as implicit cast function from the abstract to its return type. These functions are typically member-functions but they can be made \expr{static} and then serve as \tref{selective function}{types-abstract-selective-functions}.

In the example the method \expr{fromString} allows the assignment of value \expr{"3"} to variable \expr{a} of type \type{MyAbstract} while the method \expr{toArray} allows assigning that abstract to variable \expr{b} of type \type{Array<Int>}.

When using this kind of cast, calls to the cast-functions are inserted where required. This becomes obvious when looking at the \target{JavaScript} output:

\begin{lstlisting}
var a = _ImplicitCastField.MyAbstract_Impl_.fromString("3");
var b = _ImplicitCastField.MyAbstract_Impl_.toArray(a);
\end{lstlisting}
This can be further optimized by \tref{inlining}{class-field-inline} both cast functions, turning the output into the following:
\todo{please review your use of ``this'' and try to vary somewhat to avoid too much word repetition}

\begin{lstlisting}
var a = Std.parseInt("3");
var b = [a];
\end{lstlisting}
The \emph{selection algorithm} when assigning a type \expr{A} to a type \expr{B} with at least one of them being an abstract is simple:

\begin{enumerate}
	\item If \expr{A} is not an abstract, go to 3.
	\item If \expr{A} defines a \emph{to}-conversions that admits \expr{B}, go to 6.
	\item If \expr{B} is not an abstract, go to 5.
	\item If \expr{B} defines a \emph{from}-conversions that admits \expr{A}, go to 6.
	\item Stop, unification fails.
	\item Stop, unification succeeds.
\end{enumerate}

\input{assets/tikz/abstract-selection.tex}

By design, implicit casts are \emph{not transitive}, as the following example shows:

\haxe{assets/ImplicitTransitiveCast.hx}
While the individual casts from \type{A} to \type{B} and from \type{B} to \type{C} are allowed, a transitive cast from \type{A} to \type{C} is not. This is to avoid ambiguous cast-paths and retain a simple selection algorithm. 




\subsection{Operator Overloading}
\label{types-abstract-operator-overloading}

Abstracts allow overloading of unary and binary operators by adding the \expr{@:op} metadata to class fields:

\haxe{assets/AbstractOperatorOverload.hx}
By defining \expr{@:op(A * B)}, the function \expr{repeat} serves as operator method for the multiplication \expr{*} operator when the type of the left value is \type{MyAbstract} and the type of the right value is \type{Int}. The usage is shown in line 17, which turns into this when compiled to \target{JavaScript}:

\begin{lstlisting}
console.log(_AbstractOperatorOverload.
  MyAbstract_Impl_.repeat(a,3));
\end{lstlisting}
Similar to \tref{implicit casts with class fields}{types-abstract-implicit-casts}, a call to the overload method is inserted where required.

The example \expr{repeat} function is not commutative: While \expr{MyAbstract * Int} works, \expr{Int * MyAbstract} does not. If this should be allowed as well, the \expr{@:commutative} metadata can be added. If it should work \emph{only} for \expr{Int * MyAbstract}, but not for \expr{MyAbstract * Int}, the overload method can be made static, accepting \type{Int} and \type{MyAbstract} as first and second type respectively.

Overloading unary operators is analogous:

\haxe{assets/AbstractUnopOverload.hx}
Both binary and unary operator overloads can return any type.

\paragraph{Exposing underlying type operations}

It is also possible to omit the method body of a \expr{@:op} function, but only if the underlying type of the abstract allows the operation in question and if the resulting type can be assigned back to the abstract.

\haxe{assets/AbstractExposeTypeOperations.hx}

\todo{please review for correctness}


\subsection{Array Access}
\label{types-abstract-array-access}

Array access describes the particular syntax traditionally used to access the value in an array at a certain offset. This is usually only allowed with arguments of type \type{Int}. Nevertheless, with abstracts it is possible to define custom array access methods. The \tref{Haxe Standard Library}{std} uses this in its \type{Map} type, where the following two methods can be found:
\todo{You have marked ``Map'' for some reason}

\begin{lstlisting}
@:arrayAccess
public inline function get(key:K) {
  return this.get(key);
}
@:arrayAccess
public inline function arrayWrite(k:K, v:V):V {
	this.set(k, v);
	return v;
}
\end{lstlisting}
There are two kinds of array access methods:

\begin{itemize}
	\item If an \expr{@:arrayAccess} method accepts one argument, it is a getter.
	\item If an \expr{@:arrayAccess} method accepts two arguments, it is a setter.
\end{itemize}
The methods \expr{get} and \expr{arrayWrite} seen above then allow the following usage:

\haxe{assets/AbstractArrayAccess.hx}

At this point it should not be surprising to see that calls to the array access fields are inserted in the output:

\begin{lstlisting}
map.set("foo",1);
console.log(map.get("foo")); // 1
\end{lstlisting}

\paragraph{Order of array access resolving}
\label{types-abstract-array-access-order}

Due to a bug in Haxe versions before 3.2 the order of checked \expr{:arrayAccess} fields was undefined. This was fixed for Haxe 3.2 so that the fields are now consistently checked from top to bottom:

\haxe{assets/AbstractArrayAccessOrder.hx}

The array access \expr{a[0]} is resolved to the \expr{getInt1} field, leading to lower case \expr{f} being returned. The result might be different in Haxe versions before 3.2.

Fields which are defined earlier take priority even if they require an \tref{implicit cast}{types-abstract-implicit-casts}.


\subsection{Selective Functions}
\label{types-abstract-selective-functions}

Since the compiler promotes abstract member functions to static functions, it is possible to define static functions by hand and use them on an abstract instance. The semantics here are similar to those of \tref{static extensions}{lf-static-extension}, where the type of the first function argument determines for which types a function is defined:

\haxe{assets/SelectiveFunction.hx}
The method \expr{getString} of abstract \type{MyAbstract} is defined to accept a first argument of \type{MyAbstract$<$String$>$}. This causes it to be available on variable \expr{a} on line 14 (because the type of \expr{a} is \type{MyAbstract$<$String$>$}), but not on variable \expr{b} whose type is \type{MyAbstract$<$Int$>$}.

\trivia{Accidental Feature}{ Rather than having actually been designed, selective functions were discovered. After the idea was first mentioned, it required only minor adjustments in the compiler to make them work. Their discovery also lead to the introduction of multi-type abstracts, such as Map. }


\subsection{Enum abstracts}
\label{types-abstract-enum}
\since{3.1.0}

By adding the \expr{:enum} metadata to an abstract definition, that abstract can be used to define finite value sets:

\haxe{assets/AbstractEnum.hx}

The Haxe Compiler replaces all field access to the \type{HttpStatus} abstract with their values, as evident in the \target{JavaScript} output:

\begin{lstlisting}
Main.main = function() {
	var status = 404;
	var msg = Main.printStatus(status);
};
Main.printStatus = function(status) {
	switch(status) {
	case 404:
		return "Not found";
	case 405:
		return "Method not allowed";
	}
};
\end{lstlisting}

This is similar to accessing \tref{variables declared as inline}{class-field-inline}, but has several advantages:

\begin{itemize}
	\item The typer can ensure that all values of the set are typed correctly.
	\item The pattern matcher checks for \tref{exhaustiveness}{lf-pattern-matching-exhaustiveness} when \tref{matching}{lf-pattern-matching} an enum abstract.
	\item Defining fields requires less syntax.
\end{itemize}


\subsection{Forwarding abstract fields}
\label{types-abstract-forward}
\since{3.1.0}

When wrapping an underlying type, it is sometimes desirable to ``keep'' parts of its functionality. Because writing forwarding functions by hand is cumbersome, Haxe allows adding the \expr{:forward} metadata to an abstract type:

\haxe{assets/AbstractExpose.hx}

The \type{MyArray} abstract in this example wraps \type{Array}. Its \expr{:forward} metadata has two arguments which correspond to the field names to be forwarded to the underlying type. In this example, the \expr{main} method instantiates \type{MyArray} and accesses its \expr{push} and \expr{pop} methods. The commented line demonstrates that the \expr{length} field is not available.

As usual we can look at the \target{JavaScript} output to see how the code is being generated:

\begin{lstlisting}
Main.main = function() {
	var myArray = [];
	myArray.push(12);
	myArray.pop();
};
\end{lstlisting}

It is also possible to use \expr{:forward} without any arguments in order to forward all fields. Of course the Haxe Compiler still ensures that the field actually exists on the underlying type.

\trivia{Implemented as macro}{Both the \expr{:enum} and \expr{:forward} functionality were originally implemented using \tref{build macros}{macro-type-building}. While this worked nicely in non-macro code, it caused issues if these features were used from within macros. The implementation was subsequently moved to the compiler.}


\subsection{Core-type abstracts}
\label{types-abstract-core-type}

The Haxe Standard Library defines a set of basic types as core-type abstracts. They are identified by the \expr{:coreType} metadata and the lack of an underlying type declaration. These abstracts can still be understood to represent a different type. Still, that type is native to the Haxe target. 

Introducing custom core-type abstracts is rarely necessary in user code as it requires the Haxe target to be able to make sense of it. However, there could be interesting use-cases for authors of macros and new Haxe targets.

In contrast to opaque abstracts, core-type abstracts have the following properties:

\begin{itemize}
	\item They have no underlying type.
	\item They are considered nullable unless they are annotated with \expr{:notNull} metadata.
	\item They are allowed to declare \tref{array access}{types-abstract-array-access} functions without expressions.
	\item \tref{Operator overloading fields}{types-abstract-operator-overloading} that have no expression are not forced to adhere to the Haxe type semantics.
\end{itemize}



\section{Monomorph}
\label{types-monomorph}

A monomorph is a type which may, through \tref{unification}{type-system-unification}, morph into a different type later. We shall see details about this type when talking about \tref{type inference}{type-system-type-inference}.
